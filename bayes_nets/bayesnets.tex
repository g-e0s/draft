\documentclass[12pt, specialist, subf, substylefile = ../philosophy/spbu.rtx]{disser}


\usepackage[a4paper, includefoot,
            left=3cm, right=1.5cm,
            top=2cm, bottom=2cm,
            headsep=1cm, footskip=1cm]{geometry}

\usepackage[T1]{fontenc}
\usepackage[utf8]{inputenc}
\usepackage[english, russian]{babel}
\usepackage{moreverb}
\usepackage{array}

\setcounter{tocdepth}{2}

\begin{document}

\institution{}

%\title{Реферат}
\title{Bayesian Networks}

%\topic{\normalfont\scshape %
%<<Думающие машины>>}

\author{SGV}


\city{Saint-Petersburg}
\date{2018}

\maketitle


\tableofcontents


\chapter{Introduction}
McCarthy[1959]: knowledge-based of model based reasoning systems with two orthogonal components: knowledge base and reasonong engine.
In case of Probabilistic reasoning, KB is Bayesian Networks, RE is laws of probability theory


\end{document}
