\documentclass[12pt, specialist, subf, substylefile = ../philosophy/spbu.rtx]{disser}


\usepackage[a4paper, includefoot,
            left=3cm, right=1.5cm,
            top=2cm, bottom=2cm,
            headsep=1cm, footskip=1cm]{geometry}

\usepackage[T1]{fontenc}
\usepackage[utf8]{inputenc}
\usepackage[english, russian]{babel}
\usepackage{moreverb}
\usepackage{array}
\usepackage{fancyhdr}
\pagestyle{fancy}
\fancyhead{}
\fancyhead[CO]{Сарапулов Георгий}
\fancyfoot{}

\begin{document}

\section*{Отчет о проделанной работе за 2017/2018 год}
В качестве области исследования выбраны рекомендательные системы с неявной обратной связью (recommender system with implicit feedback). Изучены отличительные особенности таких систем от более распространенных систем с явной обратной связью (explicit feedback) и связанные с этим модификации в моделировании и оценке таких систем.

Изучены различные подходы к построению рекомендательных систем. На транзакционных данных розничной продуктовой сети опробованы наиболее распространенные методы, относящиеся к группе методов коллаборативной фильтрации:
\begin{itemize}
	\item на основе классификационной модели (Naive Bayes)
	\item на основе близости векторных представлений товаров (Item-to-Item neighborhood)
	\item на основе матричных разложений (Matrix Factorization)
\end{itemize}
На данный момент принято решение сконцентрироваться на байесовском подходе.

Использован байесовский подход для подбора оптимального скидочного купона, то есть такого, который в случае выдачи дает наибольший прирост к вероятности покупки рекомендуемого товара. Проведен эксперимент, подтверждающий способность данного подхода обеспечивать высокую конверсию персональных предложений. Начата подготовка публикаций по результатам эксперимента.

\end{document}