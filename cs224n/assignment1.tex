\documentclass[12pt]{article}
\usepackage{amssymb}
\usepackage{amsmath}
\usepackage{indentfirst}
\usepackage[hmarginratio=4:3, top=20mm, left=30mm, columnsep=20pt]{geometry} % Document 


\title{CS 224N: Assignment 1}

\begin{document}
\maketitle
\section{Softmax (10 points)}

(a) (5 points) Prove that softmax is invariant to constant offsets in the input, that is,
for any input vector $x$ and any constant $c$,
Applying the law of total probability we have
\begin{align*}
%P(\eta = s) &= \sum_{k=s}^{\infty} P(\eta = s | \xi = k) P(\xi = k)
%= \sum_{k=s}^{\infty} \frac{k!}{s! (k - s)!} p^s (1-p)^{k-s}  \frac{e^{-\lambda} \lambda^k}{k!} \\
%&= \frac{e^{-\lambda} \lambda^s p^s}{s!} \cdot \sum_{k=s}^{\infty} \frac{(\lambda (1 - p))^{k - s}}{(k-s)!}
softmax(\mathbf{x}) &= softmax(\mathbf{x} + c)
\end{align*}
where $x + c$ means adding the constant c to every dimension of $x$.
Remember that
\begin{align*}
softmax(\mathbf{x})_{i} &= \frac{e^{x_i}}{\sum_{j} e^{x_j}}
\end{align*}
Note: In practice, we make use of this property and choose c = − maxi xi when computing softmax
probabilities for numerical stability (i.e., subtracting its maximum element from all elements of x).


Solution:
\begin{align*}
%P(\eta = s) &= \sum_{k=s}^{\infty} P(\eta = s | \xi = k) P(\xi = k)
%= \sum_{k=s}^{\infty} \frac{k!}{s! (k - s)!} p^s (1-p)^{k-s}  \frac{e^{-\lambda} \lambda^k}{k!} \\
%&= \frac{e^{-\lambda} \lambda^s p^s}{s!} \cdot \sum_{k=s}^{\infty} \frac{(\lambda (1 - p))^{k - s}}{(k-s)!}
%softmax(\mathbf{x})_{i} &= \exp{\frac{1}{2}}
softmax(\mathbf{x} + c)_{i} &= \frac{e^{x_i + c}}{\sum_{j} e^{x_j + c}}
= \frac{e^c \cdot e^{x_i}}{e^c \cdot \sum_{j} e^{x_j}} = \frac{e^{x_i}}{\sum_{j} e^{x_j}} = softmax(\mathbf{x})_{i}
\end{align*}

\section{Neural Network basics (30 points)}
(a) (3 points) Derive the gradients of the sigmoid function and show that it can be rewritten as a function
of the function value (i.e., in some expression where only $\sigma (x)$, but not $x$, is present). Assume that the
input $x$ is a scalar for this question. Recall, the sigmoid function is
\begin{align*}
\sigma(x) &= \frac{1}{1 + e^{-x}}
\end{align*}


Solution:
\begin{align*}
\frac{\partial \sigma (x)}{\partial x} &= \frac{e^{-x}}{(1 + e^{-x})^2}
= \frac{1}{1 + e^{-x}} \cdot \frac{e^{-x}}{1 + e^{-x}} = \sigma(x) \cdot (1 - \sigma(x))
\end{align*}

(b)(3 points) Derive the gradient with regard to the inputs of a softmax function when cross entropy loss
is used for evaluation, i.e., find the gradients with respect to the softmax input vector $\theta$, when the
prediction is made by $\hat{y} = softmax(\theta)$. Remember the cross entropy function is
\begin{align*}
CE(y, \hat{y}) &= - \sum_{i} y_i log (\hat{y_i})
\end{align*}

where y is the one-hot label vector, and \^{y} is the predicted probability vector for all classes. (Hint: you
might want to consider the fact many elements of y are zeros, and assume that only the $k$-th dimension
of $y$ is one.)

Solution:
%\begin{align*}
%\frac{\partial CE(y, \hat{y})}{\partial \theta} &= \frac{\partial (- \sum_{i} y_i log (\hat{y_i}))}{\partial \theta}
%= \frac{\partial (- \sum_{i} y_i log (softmax(\theta)))}{\partial \theta}
%= \frac{\partial (- \sum_{i} y_i log (\frac{e^{\theta}}{\sum e^{\theta}}))}{\partial \theta}
%\end{align*}

\begin{align*}
\frac{\partial CE(y, \hat{y})}{\partial \theta}
= \frac{\partial CE(y, \hat{y})}{\partial \hat{y}} \cdot \frac{\partial \hat{y}}{\partial \theta} \\
\end{align*}

Derivative of the fiorst part is
\begin{align*}
\frac{\partial CE(y, \hat{y})}{\partial \hat{y}}
= \frac{\partial (- y^T log \hat{y})}{\partial \hat{y}}
= \frac{-y^T}{\hat{y}}
\end{align*}

Derivative of the second part is
\begin{align*}
\frac{\partial \hat{y}}{\partial \theta}
= \frac{\partial (\frac{e^{\theta}}{\sum_j e^{\theta}})}{\partial \theta}
= \frac{e^{\theta} \cdot (\sum_j e^{\theta} - e^{\theta})}{(\sum_j e^{\theta})^2}
= \hat{y} \cdot (1 - \hat{y})
\end{align*}

Finally
\begin{align*}
\frac{\partial CE(y, \hat{y})}{\partial \hat{y}} \cdot \frac{\partial \hat{y}}{\partial \theta}
= \frac{-y^T}{\hat{y}} \cdot \hat{y} \cdot (1 - \hat{y})
= \hat{y} \cdot (1 - \hat{y})
\end{align*}

\end{document}
