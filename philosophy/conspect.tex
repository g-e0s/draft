\documentclass[12pt, specialist, subf, substylefile = spbu.rtx]{disser}


\usepackage[a4paper, includefoot,
            left=3cm, right=1.5cm,
            top=2cm, bottom=2cm,
            headsep=1cm, footskip=1cm]{geometry}

\usepackage[T1]{fontenc}
\usepackage[utf8]{inputenc}
\usepackage[english, russian]{babel}
\usepackage{moreverb}
\usepackage{array}

\setcounter{tocdepth}{2}

\begin{document}

\institution{%
    Министерство образования и науки Российской Федерации \\
    Федеральное агентство по образованию \\
    Федеральное государственное образовательное учреждение высшего
    профессионального образования «Санкт-Петербургский государственный университет» \\
    Математико-механический факультет \\
}

%\title{Реферат}
\title{Конспект по истории и философии науки}

%\topic{\normalfont\scshape %
%<<Думающие машины>>}

\author{Сарапулов Георгий Владимирович}


\city{Санкт-Петербург}
\date{2018}

\maketitle


\tableofcontents


\chapter{История и философия науки}
\section{Основные стороны бытия науки. Характерные черты научного знания.}
Философия науки - область на границе философии и конкретного научного знания, где всеобщее, составляющее предмет философского познания, существует в неразрывном единстве с конкретным предметом научного знания. Потребность философского осмысления особенностей научного познания возникает в связи с изменениями количесва и уровня знаний в ходе исторического развития научного знания.

Науку как сложное явление необходимо рассматривать с нескольких позиций. С одной стороны, как совокупность наний и процессов их получения, то есть процессов познания. С другой стороны, наука - социальный институт, сформировавшийся на определенном этапе развития и представленный различными социальными формами организации (НИИ, университеты и т. п.). В третьих, наука является особой областью культуры и всегда находится в социально-культурном контексте.

Черты научного знания: Систематичность, Воспроизводимость, Выводимость, Доступность для обобщений и предсказаний, Проблемность, Проверяемость, Критичность, Ориентация на практику

\section{Взаимосвязь истории науки и философии науки.}
История науки и философия науки возникают и развиваются вместе с самой наукой. Объективная истоиря науки является временной последовательностью попыток построить представление о том, что такое наука. Субъективная история науки - попытки описать объективную историю науки -  зародилась позднее самой науки как ее историческое самосознание. Когда исследуются методы, применяющиеся этих попытках, изучается сам историко-научный процесс - речь идет об историографии науки.

Существует точка зрения, что история науки является дескриптивной, а философия - нормативной. Однако, согласно Юму, нормы невыводимы из фактов, а факты - из норм.

Другой подход к различению основан на оппозиции синхронического изучения науки (изучения исторических срезов научных структур), которым занимается философия, и диахронического изучения (эволюционный аспект науки), которым занимается история. Однако каждая синхроническая система имеет прошлое и будущее в качестве неотделимых структурных элементов, а эволючия носит системный характер.

 В связи с этим нет четкого разделения, в действительности обе выполняют роли интерпретации и реконструкции, взаимодействуя друг с другом.

\section{Наука и духовная культура. Функции науки в жизни общества.}

\section{Происхождение науки и периодизация истории её развития.}
В настоящее время отсутствует единое понимание происхождения науки. Признано положение, что науки вместе с философией зародилась внутри древнего мифологического сознания, но по поводу ее становления как самостоятельной области общественной деятельности есть разные точки зрения.

Наука могла возникнуть в доисторические времена с появлением первых знаний о мире и формированием продуманных навыков приспособления. Другие авторы считают временем рождения науки античность, считая основным критерием науки теоретизацию знаний против их рецептурности. Факт рождения науки связывают с учением об идеях Платона, физической теорией Аристотеля, достижения в космогонии и логике. По третьей точке зрения - в Средневековье с распространением эксперимента в естествознании. Многие считают, что наука в собственном смысле зародилась в XVI - XVII веках в период, называемый <<великой научной революцией>>, когда ученые систематически начали применять научный подход со специфическим отношением между теорией и опытом.

Таким образом, формирование науки - долгий исторический процесс

Принято выделять четыре основных периода:
\begin{enumerate}
	\item с I тыс. до н. э. до XVI века: период преднауки с осмыслением житейского опыта, натурфилософскими учениями, обособлением областей знаний
	\item XVI - XVII века: великая научная революция, когда были заложены основы современного естествознания, появились стандарты научного знания, формулировки законов в строгой математической форме, развивалась методология, появились ученые-профессионалы.
	\item XVII - XIX век: классическая наука с фундаментальными теориями в математике, естествознании, гуманитарных науках, возникновением технических наук, ростом социальной и культурной роли науки.
	\item XX век: постклассическая наука, начавшаяся научной революцией с величайшими открытиями в математике, физике, биологии, развтием нейрофизиологии, медицины, лингвистики, экономической теории, кибернетики, теории информации, характеризующаяся ростом взаимосвязей между дисциплинами и ускорением темпов изменений.
\end{enumerate}

\section{Научная революция XVI - XVII веков.}

\section{Классическая наука XVIII - XIX веков. Возникновение философии науки как особой области философского знания.}

\section{Современная наука. Историческая смена типов научной рациональности: <<классическая>>, <<неклассическая>>, <<постнеклассическая>>}

\section{Эволюция подходов к анализу науки в XX веке.}

\section{Логико-эпистемологический подход к осмыслению сущности науки.}

\section{Позитивистская традиция в философии науки.}

\section{Расширение круга философских проблем в постпозитивистской философии науки.}

\section{Философия науки в работах К.Поппера.}

\section{Теория научных революций Т.Куна.}

\section{Синтез конвенционализма и фальсификационизма в концепции философии науки И.Лакатоса.}

\section{Идея <<исследовательских традиций>> Ларри Лаудана.}

\section{<<Методологический анархизм>> П.Фейерабенда.}

\section{Социологический и культурологический подходы к исследованию развития науки.}

\section{Феноменологические и герменевтические аспекты анализа научного знания.}

\section{Наблюдение и эксперимент. Роль приборов в научном познании.}
Наблюдение - целенаправленное изучение и фиксирование данных об объекте в его естественном окружении, опирающиеся в основном на чувственные способности человека: ощущения, вопсприятия, представления. Структурные компоненты: наблюдатель, объект исследования, условия наблюдения, средства наблюдения.

Эксперимент - целенаправленное, четко выраженное активное изучение и фиксировмние данных об объекте в специально созданных и точно фиксированных и контролируемых условиях. Структурные компоненты: определенная пространственно-временная область, изуаемая система, протокол эксперимента, реакции системы.

Эксперимент оьладает рядом преимуществ: вопроизводимость, обнаружение характеристик, ненаблюдаемых в естественных условиях, возможность изолировать явление через варьирование условий, расширенные возможности использования приборов и автоматизации.

Эксперимент - связующее звено между эмпирическим и теоретическим этапами: он призван проверять определенные гипотезы, а его резултаты всегда интерпретируются с точки зрения теории. 

Роль приборов в усилении органов чувств, их дополнении новыми модальностями, повысить эффективность за счет ускорения, усиления и автоматизации некоторых мыслительных операций. Приборам свойственны погрешности, они способны вносить возмущения в наблюдаемый объект.

\section{Эмпирические факты и эмпирические зависимости. Процедуры формирования факта и его "теоретическая нагруженность".}

\section{Эмпиризм и рационализм о соотношении теории и опыта.}
Эпмирический уровень научного знания включает наблюдение, эксперимент, группировку, классификацию, описание результатов наблюдений и экспериментов, моделирование. Теоретический уровень включает в себя выдвижение, построение и разработку научных гипотез и теорий, формулирование законов, выведение следствий, сопоставление различных гипотез и теорий, процедуры объяснения и предсказания.

Уровни различаются по предмету:эмпирическое исследование направлено на явления и зависимости, теоретическое исследование начелено на выявление сущностных связей. Отличаются также средства познания, понятия, используемые методы

\section{Логическое оформление теории. Логико-методологические принципы классификации научных понятий и терминов.}

\section{<<Дилемма теоретика>> К.Гемпеля. Возможности устранения теоретических терминов (результаты Ф. Рамсея и У.Крейга)}

\section{Дедуктивная и индуктивная систематизация научной теории.}

\section{Формализация и математизация теоретического знания.}

\section{Гипотетико-дедуктивная схема развития научного познания.}

\section{Критерии выбора теории.}

\section{Философские основания науки. Роль философских идей и принципов в обосновании научного знания.}

\section{Первичные теоретические модели и законы. Принцип ceteris paribus (<<при прочих равных условиях>>).}
факторы, не входящие в вном виде в формулировку результата, остаются неизменными, эксперимент проводится в обычных условиях.

\section{Проблема включения новых теоретических представлений в культуру.}

\section{Научные революции как <<точки бифуркации>> в развитии знания. Нелинейность процесса роста знаний.}

\section{Историческое развитие институциональных форм научной деятельности.}

\section{Наука, экономика, власть. Проблемы организации, регулирования и контроля над научными исследованиями.}

\section{Главные характеристики современной  науки. Научный реализм и антиреализм.}

\section{Научный натурализм и фундаментализм.}

\section{<<Старая>> социология науки Роберта Мертона.}

\section{<<Сильная программа>> в эпистемологии науки.}

\section{Глобальный эволюционизм и современная научная картина мира.}

\section{Этика науки. Проблема ответственности учёных за их деятельность.}

\section{Сциентизм и антисциентизм.}




\chapter{Философия математики}
\section{Метафизические, семантические и эпистемологические проблемы математики}
\section{Математика как язык науки}
\section{Проблема «непостижимой» эффективности математики в естественных науках}
\section{Конвенционализм в математике}
\section{Место философии в обосновании математики}
\section{Проблема недоопределенности математической теории. Существование неизоморфных моделей}
\section{Теорема Левенгейма-Сколема, «парадокс» Сколема}
\section{Независимость континуум-гипотезы и ее отнологические последствия}
\section{Квантовый бит и квантовые вычисления}
\section{Теоремы Геделя о неполноте и их возможные философские интерпретации}
\section{Д. Лукас и Д. Хофштадтер о возможностях «мыслящих» машин и человеческого интеллекта}
\section{Математический реализм, его разновидности}
\section{Логицизм Фреге и Рассела. Неологицизм}
\section{Математический формализм. Программа Гильберта}
\section{Современный формализм Хаскеля Карри}
\section{Интуиционизм и интуиционистская логика. Алгебры Гейтинга}
\section{Возможные миры: семантика С. Крипке}
\section{Современные тенденции в философии математики}
\section{Структуралистский подход к обоснованию математики}
\section{Математическое объяснение. Математика как метафора}

\end{document}
