\documentclass[12pt, specialist, subf, substylefile = spbu.rtx]{disser}


\usepackage[a4paper, includefoot,
            left=3cm, right=1.5cm,
            top=2cm, bottom=2cm,
            headsep=1cm, footskip=1cm]{geometry}

\usepackage[T1]{fontenc}
\usepackage[utf8]{inputenc}
\usepackage[english, russian]{babel}
\usepackage{moreverb}
\usepackage{array}

\setcounter{tocdepth}{2}

\begin{document}

\institution{%
    Министерство образования и науки Российской Федерации \\
    Федеральное агентство по образованию \\
    Федеральное государственное образовательное учреждение высшего
    профессионального образования «Санкт-Петербургский государственный университет» \\
    Математико-механический факультет \\
}

%\title{Реферат}
\title{Конспект по истории и философии науки}

%\topic{\normalfont\scshape %
%<<Думающие машины>>}

\author{Сарапулов Георгий Владимирович}


\city{Санкт-Петербург}
\date{2018}

\maketitle


\tableofcontents


\chapter{История и философия науки}
\section{Основные стороны бытия науки. Характерные черты научного знания.}
Философия науки - область на границе философии и конкретного научного знания, где всеобщее, составляющее предмет философского познания, существует в неразрывном единстве с конкретным предметом научного знания. Потребность философского осмысления особенностей научного познания возникает в связи с изменениями количесва и уровня знаний в ходе исторического развития научного знания.

Науку как сложное явление необходимо рассматривать с нескольких позиций. С одной стороны, как совокупность наний и процессов их получения, то есть процессов познания. С другой стороны, наука - социальный институт, сформировавшийся на определенном этапе развития и представленный различными социальными формами организации (НИИ, университеты и т. п.). В третьих, наука является особой областью культуры и всегда находится в социально-культурном контексте.

Черты научного знания: Систематичность, Воспроизводимость, Выводимость, Доступность для обобщений и предсказаний, Проблемность, Проверяемость, Критичность, Ориентация на практику

\section{Взаимосвязь истории науки и философии науки.}
История науки и философия науки возникают и развиваются вместе с самой наукой. Объективная истоиря науки является временной последовательностью попыток построить представление о том, что такое наука. Субъективная история науки - попытки описать объективную историю науки -  зародилась позднее самой науки как ее историческое самосознание. Когда исследуются методы, применяющиеся этих попытках, изучается сам историко-научный процесс - речь идет об историографии науки.

Существует точка зрения, что история науки является дескриптивной, а философия - нормативной. Однако, согласно Юму, нормы невыводимы из фактов, а факты - из норм.

Другой подход к различению основан на оппозиции синхронического изучения науки (изучения исторических срезов научных структур), которым занимается философия, и диахронического изучения (эволюционный аспект науки), которым занимается история. Однако каждая синхроническая система имеет прошлое и будущее в качестве неотделимых структурных элементов, а эволючия носит системный характер.

 В связи с этим нет четкого разделения, в действительности обе выполняют роли интерпретации и реконструкции, взаимодействуя друг с другом.

\section{Наука и духовная культура. Функции науки в жизни общества.}
Роль науки в духовной жизни легко проследить по специальным дисциплинам, изучающим науку на общественный институт: социология науки, психология науки, политология науки, экономика науки. Как общественному феномену науке свойственны четыре основополагающие характеристики: наука является социальным институтом, организованным по специфическим правилам; сообществом ученых, объединенных на основе определенных внутренних установок; играет весомую политическую роль и влияет на экономическую модель общества.

Положение науки в обществе - результат процесса развития научного мышления, начавшегося задолго до выделения науки в отдельный социальный институт. Еще в античности человек начинает выводить законы на основе рациональной аргументации, не только опираясь на веру и традиции. В средневековье появляются университеты, получившие признание на государственном уровне. В новое время наука становится профессиональной деятельностью с требованием наличия комплекса навыков для тех. кто хотел заниматься исследованиями. В XX веке наука становится коллективным проектом. К концу века исследования начинают регулироваться экономической целесообразностью и првктической применимостью и выступает катализатором политических процессов. Сегодня наука - структурообразующий интитут в общественном организме.

\section{Происхождение науки и периодизация истории её развития.}
В настоящее время отсутствует единое понимание происхождения науки. Признано положение, что науки вместе с философией зародилась внутри древнего мифологического сознания, но по поводу ее становления как самостоятельной области общественной деятельности есть разные точки зрения.

Наука могла возникнуть в доисторические времена с появлением первых знаний о мире и формированием продуманных навыков приспособления. Другие авторы считают временем рождения науки античность, считая основным критерием науки теоретизацию знаний против их рецептурности. Факт рождения науки связывают с учением об идеях Платона, физической теорией Аристотеля, достижения в космогонии и логике. По третьей точке зрения - в Средневековье с распространением эксперимента в естествознании. Многие считают, что наука в собственном смысле зародилась в XVI - XVII веках в период, называемый <<великой научной революцией>>, когда ученые систематически начали применять научный подход со специфическим отношением между теорией и опытом.

Таким образом, формирование науки - долгий исторический процесс

Принято выделять четыре основных периода:
\begin{enumerate}
	\item с I тыс. до н. э. до XVI века: период преднауки с осмыслением житейского опыта, натурфилософскими учениями, обособлением областей знаний
	\item XVI - XVII века: великая научная революция, когда были заложены основы современного естествознания, появились стандарты научного знания, формулировки законов в строгой математической форме, развивалась методология, появились ученые-профессионалы.
	\item XVII - XIX век: классическая наука с фундаментальными теориями в математике, естествознании, гуманитарных науках, возникновением технических наук, ростом социальной и культурной роли науки.
	\item XX век: постклассическая наука, начавшаяся научной революцией с величайшими открытиями в математике, физике, биологии, развтием нейрофизиологии, медицины, лингвистики, экономической теории, кибернетики, теории информации, характеризующаяся ростом взаимосвязей между дисциплинами и ускорением темпов изменений.
\end{enumerate}

\section{Научная революция XVI - XVII веков.}
Происходило постепенное отождествление абстрактно-математической и реалистской установок. Началом научной революции принято считать с трактат Коперника <<О вращени небесных сфер>> (1543). Кеплер создает космологию, основанную на соответствии между физической реальностью и математической моделью. У Ньютона - отождествление математического пространства с реальным и абсолютным.
Закрепляется идея тождественности механизма и физическим бытием. Происходит пересмотр античных предпосылок

\section{Классическая наука XVIII - XIX веков. Возникновение философии науки как особой области философского знания.}
Важная черта эпохи - уверенность в универсальности разума, исследовательская программа основана на определенных свойствах науки: идеале механицистского исследования природы и представлении о прогрессе научного знания.
Биология и зоология, подобно медицине, рассматривались как знание, направленное на формирование тела, вследствие - и души. С другой стороны - части натурфилософии, всеобщей философии природы или физиологии. С оформлением биологии связана парадигма естествознания, а затем и социальных наук - стремление к достижению гармонии и единства научных дисциплин и обоснованию пути науки к моральной значимости, отождествлению науки и этики, знания и блага.

Материя понималась как пассивная, однородная, сводимая к геометрическому протяжению. Трансформируется взгляд на пространство и материю, возникла концепция притяжения - того, чего не видно, но что постоянно действует.

Возникает скептицизм, видящий своим идеалом не в формировании математических абстракций, а в сборе уникальных фактов. Оппонировал скептицизму неомеханицизм (инструментализм), охвативший математику и физику, к рамках которого математика рассматривалась как инструмент познания, а не истинная модель реальности, вопросы о сущности материи  или поределении силы признавались слишком неясными. Развивалась вероятностная математика, воспринимаемая совершенным инструментом для управления наблюдающим разумом.

В спорах в области физики и наук о жизни приобрели смысл, вкладываемый сегодня в термин <<философия науки>>. 

\section{Современная наука. Историческая смена типов научной рациональности: <<классическая>>, <<неклассическая>>, <<постнеклассическая>>}
Отличия научной рациональности от философской в том, что наука не включает в себя условия своего обоснования, но при этом обязательно непротиворечива. Философия, напротив, ялвяется полной системой в смысле самообоснования, но жертвует непротиворечивостью, тем не менее включая ее в качестве творческого источника жизни всей системы.
Согласно Степину, выделяются другие три типа: классический (элиминация субъекта), неклассический (связи между знаниями и характером средств и операций деятельности) и постнеклассический (рефлексия над деятельностью, ценностно-целевыми структурами).

Современная рациональность включает критико-рефлексивную установку по отношению к собственным предпосылкам, и предметом рационального сознания становится деятельность по выработке рационального знания на основе познавательных средств и предпосылок. Она возникает с пониманием невозможности универсального единого описания мира, в едином порядке обнаруживаются противоречия, приходит понимание, что наблюдаемое зависит от процесса наблюдения.

\section{Эволюция подходов к анализу науки в XX веке.}

\section{Логико-эпистемологический подход к осмыслению сущности науки.}

\section{Позитивистская традиция в философии науки.}
Позитивизм складывался под лозунгом борьбы с умозрительной философией, полагая метафизику излишней, за исключением ее роли как метанауки. По мнению основателя позитивизма О. Конта, преодоление умозрительного характера - неизбежное следствие взросления разума.

Человеческое мышление как организм проходит три стадии: теологическую или фиктивную (разум в примитивной потребности ищет причины всего, порождает причины, отсутствующие в природе), метафизическую или абстрактную (мышление объясняет то, чего никогда не существовало: бытие, сущность) и реальную или положительную (подчинение воображения наблюдению, критическая ревизия метафизических понятий, удаление бессмысленных вопросов).

Цель позитивного мышления - удовлетворение собственных потребностей.
Для Спенсера наука - расширенный и усложненный здравый смысл. Наиболее известная идея - концепция социального организма: общество - живой организм, страты выполняют различные функции.

Мах основывался на трех принципах: эпистемологический принцип <<экономии мышления>> (предвосхищение фактов в мысли, абстрактность, удаление объяснительной части и метафизических категорий, замена причинности функцией), гносеологический принцип нерасчлененности субъекта и объекта (тождество психического и физического, субъективного и объективного; цвета, тоны, давления - настоящие элементы мира, не вещи; научное знание - в описании функциональных связей с помощью численных величин) и принцип конвенциональной природы научной теории (математика - лишь конвенциональный инструмент)

\section{Расширение круга философских проблем в постпозитивистской философии науки.}
Ведущий методологический регулятив - релятивизм: научная теория есть результат переплетения различных процессовв обществе, организующих мышление ученого. Центральную роль играет полемика с кумулятивной моделью развития научного знания, исходившей из принципов: неподверженность сомнению однажды обоснованных теорий, отказ от пересмотра истории науки, неизменность накопленного запаса правильных знаний, неизбежность нахождения и безвозвратного удаления заблуждений, отграничение науки от ненаучных форм знания.

Из принципов сформировалась проблематика постпозитивизма: отбор и интерпретация фактов, соотношение теории и эмпирии (влияние теории на отбор фактов), проблема интернализма-экстернализма (влияние процессов в обществе на содержание науки).


\section{Философия науки в работах К.Поппера.}
Верификация не должна быть единственным методологическим ориентиром научного исследования, так как верифицировать можно что угодно. Индуктивный переход не гарантирует истинности. Решить обе проблемы можно, отказавшись от ллогического перехода от фактов к теории - факт внелогичен, логично мышление. Интерпретации и осмысления, возникшие независимо от фактов, пригодны для формулирования теории, то есть теория строится мышлением, черпающем материал в самом себе. Решение возможного роста числа вариантов теорий по одному набору фактов - выдвижение только фальсифицируемых теорий. То есть теория должна бытьлогически непротиворечивой и предполагать факты, которые опровергнут теорию в случае и обнаружения. Таким образом убирается переход от опыта к теории, из теории логически выводятся факты, способные ее фальсифицировать. Метод проб и ошибок - наиболее рациональная процедура. Эти взгляды строго отделяют научное знание от ненаучного.

\section{Теория научных революций Т.Куна.}
В <<Структуре научных революций>> расматривал вопрос выживаемости научных концепций, видя причиной смены не новые факты, а пересмотр отношения научного сообщества к аномалиям. Между сменяющимися парадигмами может не вестись логическая дискуссия, используются различные трактовки одних и тех же терминов, побеждает претендент, наиболее подходящий научному сообществу в данный момент. 
Следствие образования первой парадигмы: прекращение научных дискуссий по фундаментальным проблемам, исчезновение научных школ с иными взглядами, организуется научное образование соответственно парадигме, где принципы преподаются как подтвержденные и единственно возможные, облегчается труд ученого, так как он заранее знает, ка котобрать актуальные факты и как строить (уточнять) теорию, прекращаетсясоздание фундаментальных трудов и наука становится узкоспециализированной.
После смены парадигмы - новая картина мира признается единственно верной, преподавание наук переориентируется, переписываются учебники по истории науки.
Нормальная наука развивается кумулятивно через гипотетико-дедуктивный метод до следующего кризиса, при котором либо приспособится к аномалиям, либо переживет революцию.

Теория фрагмента реальности возникает после выделение одного объяснения случайно накопленных фактов по причинам социально-психологического характера, не более высокой степени мпирической подтвержденности. Парадигма формируется при охватывании всех известных областей реальности исходящими из одних фундаментальных принципов теорий. Она имеет два измерения: метафизическое (принципы организации мира) и социальное (то, что объединяет членов научного сообщества, эквивалентно ему). 

В <<Дополнении 1969 года>> выделяет четыре компонена: символические обобщения (облеченные в логическую форму выражения), метафизические части парадигмы (убеждение в истинности моделей), методологические ценности (критерий научности) и образцы (конкретные способы решения конкретных проблем)



\section{Синтез конвенционализма и фальсификационизма в концепции философии науки И.Лакатоса.}
Твердое ядро теории (ряд связанных высказываний и формул, внятно выражающих основную идею теории), отрицательная (запрет изменения ядра теории) и положительная (возможность изменений теории с сохранением ядра) эвристики формируют научно-исследовательскую программу.
Сменить теорию может независимая от первой теория-конкурент, которая имеет абсолютно отличное ядро, обладает отрицательной эвристикой (одинаковая для всех теорий), иной положительной эвристикой и более мощной эмпирической базой и эвристической силой.
Механизм фальсификации через другую теорию с иной концепцией реальности и подтвержлдаемой большим множеством фактов, в том числе фактами, подтверждавшими фальсифицируемую теорию.


\section{Идея <<исследовательских традиций>> Ларри Лаудана.}

\section{<<Методологический анархизм>> П.Фейерабенда.}
<<Против методологического принуждения>>,  <<Наука в свободном обществе>>

Пытался преодолеть объективность процесса формирования и смены научных теорий и ее негативные последствия. Избежать догматизма и застоя можно, устранив причину - отсутствие конкурирующих теорий. Принцип пролиферации - необходимость создания альтернативной теории с принципиально иным твердым ядром по Лакатосу через изобретение новой концептуальной системы, несовместимой с наиболее обоснованными результатами наблюдения и нарушающая самые правдоподобные теоретические принципы. не обращая внимания на трудности (принцип упорства). По принципу изменения значения (собственный язык в новой теории), принципу несоизмеримости (невозможность сравнительного анализа теорий) и принципу вседоступности допускаются любые концепции и теории и запрещается взаимная критика.

Наука должна быть отделена от государства, так как конкурентную борьбу выигрывает созвучная социально-политической обстановке теория.

\section{Социологический и культурологический подходы к исследованию развития науки.}

\section{Феноменологические и герменевтические аспекты анализа научного знания.}

\section{Наблюдение и эксперимент. Роль приборов в научном познании.}
Наблюдение - целенаправленное изучение и фиксирование данных об объекте в его естественном окружении, опирающиеся в основном на чувственные способности человека: ощущения, вопсприятия, представления. Структурные компоненты: наблюдатель, объект исследования, условия наблюдения, средства наблюдения.

Эксперимент - целенаправленное, четко выраженное активное изучение и фиксировмние данных об объекте в специально созданных и точно фиксированных и контролируемых условиях. Структурные компоненты: определенная пространственно-временная область, изуаемая система, протокол эксперимента, реакции системы.

Эксперимент оьладает рядом преимуществ: вопроизводимость, обнаружение характеристик, ненаблюдаемых в естественных условиях, возможность изолировать явление через варьирование условий, расширенные возможности использования приборов и автоматизации.

Эксперимент - связующее звено между эмпирическим и теоретическим этапами: он призван проверять определенные гипотезы, а его резултаты всегда интерпретируются с точки зрения теории. 

Роль приборов в усилении органов чувств, их дополнении новыми модальностями, повысить эффективность за счет ускорения, усиления и автоматизации некоторых мыслительных операций. Приборам свойственны погрешности, они способны вносить возмущения в наблюдаемый объект.

\section{Эмпирические факты и эмпирические зависимости. Процедуры формирования факта и его "теоретическая нагруженность".}

Теория погружена в интертеоретический контекст, специально сформулированные подтеории служат для сопоставления с опытом. Проверка теории проходит четыре ступени: метатеоретическую (непротиворечивость, недвусмысленность содержания, проверяемость), интертеоретическую (совместимость с другими теориями), философскую (оценка достоинств в свете философской концепции) и эмпирическую. 

\section{Эмпиризм и рационализм о соотношении теории и опыта.}
Эпмирический уровень научного знания включает наблюдение, эксперимент, группировку, классификацию, описание результатов наблюдений и экспериментов, моделирование. Теоретический уровень включает в себя выдвижение, построение и разработку научных гипотез и теорий, формулирование законов, выведение следствий, сопоставление различных гипотез и теорий, процедуры объяснения и предсказания.

Уровни различаются по предмету:эмпирическое исследование направлено на явления и зависимости, теоретическое исследование начелено на выявление сущностных связей. Отличаются также средства познания, понятия, используемые методы.

Дилемма рационализма и эмпиризма связана с убеждением, что философия Нового времени является по преимуществу эпистемологией, редуцирующей философские вопросы к поиску надежного основания познания. Рационализм решающую роль в познании приписывал разуму, эмпиризм - опыту. Эпистемологический характер теорий постулирует разрыв между субъектом познания и внешним миром.

\section{Логическое оформление теории. Логико-методологические принципы классификации научных понятий и терминов.}

\section{<<Дилемма теоретика>> К.Гемпеля. Возможности устранения теоретических терминов (результаты Ф. Рамсея и У.Крейга)}

\section{Дедуктивная и индуктивная систематизация научной теории.}

\section{Формализация и математизация теоретического знания.}
Формализация - методы познания, состоящии в отвлечении от содержания знания об объекте, понятий и других форм мышления, посредством которых выражено знание об объекте на естественном языке науки с дальнейшим исследованием объекта через изучение формы знания о нем, представленного в специальном, формализованном языке. Для возможности изучения формы знания, следующей за содержанием, требуется выявление и уточнение ее элементов и связей - эта уточненная форма и изучается при использовании метода формализации.

Структура: символизация (перевод на формализованный язык) - преобразование (операции по формальным правилам) - интерпретация (истолкование результатов на естественном языке) + практическая проверка

Стандарты: непротиворечивость представления, корректность (истинность результата), адекватность (выводимость истинного в формализованном представлении)

Достоинства: четкое выделение и представление предположений, возможность математической проверки и моделирования, выход в случае сложности формулировки на естественном языке.

\section{Гипотетико-дедуктивная схема развития научного познания.}
Основная идея гипотетико-дедктивной схемы - выдвижение определенных гипотез и последующая проверка непосредственно проверяемых следствий из них. Как правило имеют дело с совокупностью гипотез различного уровня, системы с дедуктивными связями между уровнями, где более низкий уровень соответствует следствиям из более высоких.

Выдвигаемая гипотеза основывается на данных опыта, так что описания опытных данных выводятся из нее, подвергается проверке. Опровержение не является основанием для отказа от гипотезы, но побуждением к дальнейшему исследованию.

\section{Критерии выбора теории.}
Теория должна быть принципиально проверяемой (возможность опровержения, отсутствие совместности с любым исходом опыта), максимально общной (из теории выводится широкая совокупность описаний и предсказаний), системной (нахождение в парадигме и целостность), обладать предсказательной силой и принципиальной простотой (минимум допущений, не связанных с предположениями)

\section{Философские основания науки. Роль философских идей и принципов в обосновании научного знания.}
Под основаниями науки понимают систему регулятивов, определяющих цель и способы получения научного знания, представление и понимание научной реальности, формы и степень обоснованности научного знания и его включения в культуру. Цель и способы определяются идеалами, нормами и критериями, обобщенное понимание и представление воплощается в научной картине мира, формы и степень обоснованности науки и ее включения в культурный контекст обеспечивают философские основания.

К логико-эпистемологическим номативам науки относят описание (выявление совокупности данных о свойствах и отношениях), объяснение (выработка понимания сущности возникновения, развития, функционирования), системность (анализ и соотнесение данных по типам и классам, введение новых типов и классов), доказательность и обоснованность (соответствие логике), эвристичность (способность предсказывать новые свойства и отношения, открытие новых уровней организации мира и новых типов объектов).

К социокультурным нормам относят прагматическую (способы применения знания), прогностическую (перспективы развития, футурологические модели, рекомендации на будущее), экспертную (анализ и оценка проектов и программ).

Научная картина мира как возникшая на основе обобщения и синтеза основных фактов, понятий и принципов система представлений о фундаментальных свойствах и закономерностях универсума,состоит из концептуального (философские принципы и категории, общенаучные положения и понятия) и чувственного (наглядные представления в виде моделей) компонентов. Современная научная картина мира состоит из естественнонаучного, технического и социально-гуманитарного блоков.

Можно выделить две подсистемы философских оснований научного знания: онтологическую (сеть категорий для задания понимания реальности) и эпистемологическую (категориальные схемы для описания процедур и результатов)

\section{Первичные теоретические модели и законы. Принцип ceteris paribus (<<при прочих равных условиях>>).}
факторы, не входящие в вном виде в формулировку результата, остаются неизменными, эксперимент проводится в обычных условиях.

\section{Проблема включения новых теоретических представлений в культуру.}
Новые теоретические представления в современной постиндустриальной культуре включаются в нее путем постоянной перестройки ее фундаментальных положений, влекущих перестройку социальной и духовной сфер, таким образом затрагиваются интересы каждого человека. В первую очередь новые представления касаются метафизических положений, и у них могут возникнуть сложности с традиционными представлениями, ставшими частью обыденного опыта. Любое новое теоретическое представление должно пройти сложную работу, иногда изменяющую его до неузнаваемости, несколько этапов цензуры (гласной и негласной, сознателной и бессознательной), в чем должно помогать образование, одной из важнейших функций которого становится производство социально востребованного типа личности.

\section{Научные революции как <<точки бифуркации>> в развитии знания. Нелинейность процесса роста знаний.}
Революции выступают как разрешение многогранного противоречия между старым и новым знанием, наличие двух фаз - эволюционной и революционной - выражение принципиальной нелинейности процесса роста знаний. Часто наука напоминает движение вспять, когда новые теории формулируются на основе ранееотброшенных идей. Революции вызываются ростом числа фактов, не поддающихся объяснению и связанной с этим необходимостью выработки новых теоретических представлений, кардинальной перестройкой картины мира и философским обоснованием новаций, включающим их в общекультруный фон.

Выделяют четыре типа научных революций: глобальную (вся наука), комплексную (ряд областей), частную (одна область) и научно-техническую (преобразование производительных сил).

\section{Историческое развитие институциональных форм научной деятельности.}

\section{Наука, экономика, власть. Проблемы организации, регулирования и контроля над научными исследованиями.}

\section{Главные характеристики современной  науки. Научный реализм и антиреализм.}
Фундамент эволюционно-синергетической научной картины мира составляют синергетика, системология и информационный подход, в рамках которого информация понимается как атрибут материи наряду с движением, пространством и временем. Развитие расматривается как универсальный и глобальный самодетерминированный нелинейный процесс самоорганизации нестационарныхоткрытых систем. Утверждается фундаментальная могласованность основных законов Вселенной с существованием в ней жизни и разума.

В общенаучной концепции универсального эволюционизма принцип развития воспроизводится на уровне оснований науки, служащей центром идейной кристаллизации новой научной картины мира. Элиминируется понятие изолированной системы и концепции абсолютного детерминизма, всякий локальный процесс эволюции объясняется толькокак необходимый момент единого универсального процесса развития Вселенной как целого. Произошло осознание целостности и системности Метагалактики, доступной научному познанию части мира. Развитие трактуется как нелинейный, вероятностный и необратимый процесс. Одно из центральных мест занимает антропный принцип, согласно которому человеческое бытие рассматривается как эндогенная форма бытия по отношению к миру и природе (<<Мир таков, потому что существует человек>>). Протекание эволюционных процессов представляется в форме самоорганизации сложных систем, обнаруживающих общие закономерности.

Научный реализм — течение в философии науки, согласно которому единственным надёжным средством достижения знания о мире является научное исследование, результат которого интерпретируется с помощью научных теорий. Теории научного реализма могут быть также вероятно истинными или приблизительно истинными или относительно истинными. Теории касаются наблюдаемых и ненаблюдаемых объектов, хотя и являются в сущности достоверными, однако могут быть в какой-то степени ложны. Исследуемые объекты независимы от нашего разума, а научные теории достоверны по отношению к внешнему (объективному) миру. Задача научного реализма — сформулировать истинные утверждения о реальности, что производится в сопровождении лучших научных теорий. Понятие реализма в научных терминах полезнее рассмотреть в трех измерениях:

Семантический реализм полагает, что теории являются истинными либо ложными описаниями реальности. Истинность и ложность зависит от того, существуют ли данные объекты и насколько верно они описаны теорией.
Метафизический реализм предполагает существование реальности независимо от нашего знания. Теория должна отвечать <<метафизическим представлениям>>. По словам Патнэма: <<метафизический реалист предлагает нам принять некоторую картину так, словно эта картина сама себя объясняет>>.
Эпистемологический реализм полагает, что доверие суждениям об истинности теории по преимуществу оправдано. Эпистемологический научный реалист считает, что наука сопровождается лучшими научными теориями, даже если они не могут быть доказаны с абсолютной уверенностью; можно предположить, что некоторые теории могут оказаться значительно ошибочными, тем не менее научный реалист уверен в том, что они в какой-то степени истинны.

Антиреализм утверждает, что наблюдение как основа научного знания является теоретически нагруженным, т. е. из него в принципе невозможно исключить некоторый элемент субъективности. Мы изменяем саму физическую реальность, когда пытаемся с ней взаимодействовать, чтобы изучить ее, а также теоретически нагружаем каждый акт наблюдения, интерпретируя его с позиции базового, принятого нами ранее знания.

\section{Научный натурализм и фундаментализм.}

\section{<<Старая>> социология науки Роберта Мертона.}

\section{<<Сильная программа>> в эпистемологии науки.}

\section{Глобальный эволюционизм и современная научная картина мира.}
Научная картина мира - синтетическое, систематизированное и целостное представление о природе на данном этапе научного познания. Ее эволюционность соответствует эволюционному характеру научного знания как такового, ее интегративность в том, что она является центром собирания, систематизации и согласования данных отдельных науксдля создания целостного образа мира.

Эволюция научной картины мира прошла этапы космоцентризма, теоцентризма, механистической картины, вероятностной картины и, наконец, информационной (информация - основная форма обобщения и передачи знания).

\section{Этика науки. Проблема ответственности учёных за их деятельность.}
Выделяются этически нейтральные и этически оцениваемые деяния и объекты.
Этика научной дискуссии состоит в контроле за отсутствием сознательных логических ошибок при аргументации, отказе от использования способов доказательства, при помощи которых можно доказать все что угодно (апеллирование к интуиции, ограниченности человеческого разума и т. п.), проведении четкой границы между научной позицией и личными качествами собеседника.

Этика публикаций заключается в публикации исключительно своих идей (обязательны ссылки на использованные работы других авторов), донесение в том числе отрицательных результатов, использовании для размещения специализированных научных журналов.

Мертон выделяет четыре регулятива научной деятельности: универсализм (предположение, что изучаемые явления везде протекают одинаково), коллективизм (научное знание должно быть достоянием всего научного сообщества), бескорыстность (главный стимул - поиск истины) и организованный скептицизм (перепроверка заимствованных данных, отказ от несостоятельных идей).

Социальная этика науки призвана ответить на вопрос, чем должен определяться научный прогресс: объективной логикой развития науки или социальной ответственностью ученого. В связи с этим возникают вопросы о том, кто несет ответственность за негативное использование результатов и о необходимости прекращать исследование, если последствия его использования наверняка окажутся деструктивными.

\section{Сциентизм и антисциентизм.}
Сциентизм базируется на абсолютизации рационально-теоретических компонентов знания. Он отстаивает позицию, согласно которой только научное знание является настоящим знанием, что методы и допущения (включая эпистемологические и метафизические учения), на которых основаны естественные науки, могут быть использованы в общественных и гуманитарных науках. Характерна преувеличенная вера в науку как средство получения знаний и решения стоящих перед человечеством проблем и приверженность натуралистической, материалистической, преимущественно механистической метафизике. Именно эта приверженность является ключевой особенностью сциентизма, и именно она служит главным объектом критики, поскольку сциентизм не просто отстаивает мощь науки, но выдвигает метафизические требования.

Антисциентизм (Гуссерль, Виндельбанд, Риккерт) опирается на ключевуюроль этических, правовых, культурных ценностей по отношению к идеалу научности. При любой рациональной доктрине не удастся освободиться от изначальной субъективности и влияния культурного контекста, от него отчужден комплекс научного знания, он не несет ответственности за применение своих открытий и не задумывается об их влиянии.



\chapter{Философия математики}
\section{Метафизические, семантические и эпистемологические проблемы математики}
\section{Математика как язык науки}
\section{Проблема «непостижимой» эффективности математики в естественных науках}
\section{Конвенционализм в математике}
\section{Место философии в обосновании математики}
\section{Проблема недоопределенности математической теории. Существование неизоморфных моделей}
\section{Теорема Левенгейма-Сколема, «парадокс» Сколема}
\section{Независимость континуум-гипотезы и ее отнологические последствия}
\section{Квантовый бит и квантовые вычисления}
\section{Теоремы Геделя о неполноте и их возможные философские интерпретации}
\section{Д. Лукас и Д. Хофштадтер о возможностях «мыслящих» машин и человеческого интеллекта}
\section{Математический реализм, его разновидности}
\section{Логицизм Фреге и Рассела. Неологицизм}
\section{Математический формализм. Программа Гильберта}
\section{Современный формализм Хаскеля Карри}
\section{Интуиционизм и интуиционистская логика. Алгебры Гейтинга}
\section{Возможные миры: семантика С. Крипке}
\section{Современные тенденции в философии математики}
\section{Структуралистский подход к обоснованию математики}
\section{Математическое объяснение. Математика как метафора}

\end{document}
