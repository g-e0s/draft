\documentclass[12pt, specialist, subf, substylefile = spbu.rtx]{disser}


\usepackage[a4paper, includefoot,
            left=3cm, right=1.5cm,
            top=2cm, bottom=2cm,
            headsep=1cm, footskip=1cm]{geometry}

\usepackage[T1]{fontenc}
\usepackage[utf8]{inputenc}
\usepackage[english, russian]{babel}
\usepackage{moreverb}
\usepackage{array}

\setcounter{tocdepth}{2}

\begin{document}

\institution{%
    Министерство образования и науки Российской Федерации \\
    Федеральное агентство по образованию \\
    Федеральное государственное образовательное учреждение высшего
    профессионального образования «Санкт-Петербургский государственный университет» \\
    Математико-механический факультет \\
}

%\title{Реферат}
\title{Конспект по истории и философии науки}

%\topic{\normalfont\scshape %
%<<Думающие машины>>}

\author{Сарапулов Георгий Владимирович}


\city{Санкт-Петербург}
\date{2018}

\maketitle


\tableofcontents


\chapter{История и философия науки}
\section{Основные стороны бытия науки. Характерные черты научного знания.}
Философия науки - область на границе философии и конкретного научного знания, где всеобщее, составляющее предмет философского познания, существует в неразрывном единстве с конкретным предметом научного знания. Потребность философского осмысления особенностей научного познания возникает в связи с изменениями количесва и уровня знаний в ходе исторического развития научного знания.

Науку как сложное явление необходимо рассматривать с нескольких позиций. С одной стороны, как совокупность наний и процессов их получения, то есть процессов познания. С другой стороны, наука - социальный институт, сформировавшийся на определенном этапе развития и представленный различными социальными формами организации (НИИ, университеты и т. п.). В третьих, наука является особой областью культуры и всегда находится в социально-культурном контексте.

Черты научного знания: Систематичность, Воспроизводимость, Выводимость, Доступность для обобщений и предсказаний, Проблемность, Проверяемость, Критичность, Ориентация на практику

\section{Взаимосвязь истории науки и философии науки.}
История науки и философия науки возникают и развиваются вместе с самой наукой. Объективная истоиря науки является временной последовательностью попыток построить представление о том, что такое наука. Субъективная история науки - попытки описать объективную историю науки -  зародилась позднее самой науки как ее историческое самосознание. Когда исследуются методы, применяющиеся этих попытках, изучается сам историко-научный процесс - речь идет об историографии науки.

Существует точка зрения, что история науки является дескриптивной, а философия - нормативной. Однако, согласно Юму, нормы невыводимы из фактов, а факты - из норм.

Другой подход к различению основан на оппозиции синхронического изучения науки (изучения исторических срезов научных структур), которым занимается философия, и диахронического изучения (эволюционный аспект науки), которым занимается история. Однако каждая синхроническая система имеет прошлое и будущее в качестве неотделимых структурных элементов, а эволючия носит системный характер.

 В связи с этим нет четкого разделения, в действительности обе выполняют роли интерпретации и реконструкции, взаимодействуя друг с другом.

\section{Наука и духовная культура. Функции науки в жизни общества.}
Роль науки в духовной жизни легко проследить по специальным дисциплинам, изучающим науку на общественный институт: социология науки, психология науки, политология науки, экономика науки. Как общественному феномену науке свойственны четыре основополагающие характеристики: наука является социальным институтом, организованным по специфическим правилам; сообществом ученых, объединенных на основе определенных внутренних установок; играет весомую политическую роль и влияет на экономическую модель общества.

Положение науки в обществе - результат процесса развития научного мышления, начавшегося задолго до выделения науки в отдельный социальный институт. Еще в античности человек начинает выводить законы на основе рациональной аргументации, не только опираясь на веру и традиции. В средневековье появляются университеты, получившие признание на государственном уровне. В новое время наука становится профессиональной деятельностью с требованием наличия комплекса навыков для тех. кто хотел заниматься исследованиями. В XX веке наука становится коллективным проектом. К концу века исследования начинают регулироваться экономической целесообразностью и првктической применимостью и выступает катализатором политических процессов. Сегодня наука - структурообразующий интитут в общественном организме.

\section{Происхождение науки и периодизация истории её развития.}
В настоящее время отсутствует единое понимание происхождения науки. Признано положение, что науки вместе с философией зародилась внутри древнего мифологического сознания, но по поводу ее становления как самостоятельной области общественной деятельности есть разные точки зрения.

Наука могла возникнуть в доисторические времена с появлением первых знаний о мире и формированием продуманных навыков приспособления. Другие авторы считают временем рождения науки античность, считая основным критерием науки теоретизацию знаний против их рецептурности. Факт рождения науки связывают с учением об идеях Платона, физической теорией Аристотеля, достижения в космогонии и логике. По третьей точке зрения - в Средневековье с распространением эксперимента в естествознании. Многие считают, что наука в собственном смысле зародилась в XVI - XVII веках в период, называемый <<великой научной революцией>>, когда ученые систематически начали применять научный подход со специфическим отношением между теорией и опытом.

Таким образом, формирование науки - долгий исторический процесс

Принято выделять четыре основных периода:
\begin{enumerate}
	\item с I тыс. до н. э. до XVI века: период преднауки с осмыслением житейского опыта, натурфилософскими учениями, обособлением областей знаний
	\item XVI - XVII века: великая научная революция, когда были заложены основы современного естествознания, появились стандарты научного знания, формулировки законов в строгой математической форме, развивалась методология, появились ученые-профессионалы.
	\item XVII - XIX век: классическая наука с фундаментальными теориями в математике, естествознании, гуманитарных науках, возникновением технических наук, ростом социальной и культурной роли науки.
	\item XX век: постклассическая наука, начавшаяся научной революцией с величайшими открытиями в математике, физике, биологии, развтием нейрофизиологии, медицины, лингвистики, экономической теории, кибернетики, теории информации, характеризующаяся ростом взаимосвязей между дисциплинами и ускорением темпов изменений.
\end{enumerate}

\section{Научная революция XVI - XVII веков.}
Происходило постепенное отождествление абстрактно-математической и реалистской установок. Началом научной революции принято считать с трактат Коперника <<О вращени небесных сфер>> (1543). Кеплер создает космологию, основанную на соответствии между физической реальностью и математической моделью. У Ньютона - отождествление математического пространства с реальным и абсолютным.
Закрепляется идея тождественности механизма и физическим бытием. Происходит пересмотр античных предпосылок

\section{Классическая наука XVIII - XIX веков. Возникновение философии науки как особой области философского знания.}
Важная черта эпохи - уверенность в универсальности разума, исследовательская программа основана на определенных свойствах науки: идеале механицистского исследования природы и представлении о прогрессе научного знания.
Биология и зоология, подобно медицине, рассматривались как знание, направленное на формирование тела, вследствие - и души. С другой стороны - части натурфилософии, всеобщей философии природы или физиологии. С оформлением биологии связана парадигма естествознания, а затем и социальных наук - стремление к достижению гармонии и единства научных дисциплин и обоснованию пути науки к моральной значимости, отождествлению науки и этики, знания и блага.

Материя понималась как пассивная, однородная, сводимая к геометрическому протяжению. Трансформируется взгляд на пространство и материю, возникла концепция притяжения - того, чего не видно, но что постоянно действует.

Возникает скептицизм, видящий своим идеалом не в формировании математических абстракций, а в сборе уникальных фактов. Оппонировал скептицизму неомеханицизм (инструментализм), охвативший математику и физику, к рамках которого математика рассматривалась как инструмент познания, а не истинная модель реальности, вопросы о сущности материи  или поределении силы признавались слишком неясными. Развивалась вероятностная математика, воспринимаемая совершенным инструментом для управления наблюдающим разумом.

В спорах в области физики и наук о жизни приобрели смысл, вкладываемый сегодня в термин <<философия науки>>. 

\section{Современная наука. Историческая смена типов научной рациональности: <<классическая>>, <<неклассическая>>, <<постнеклассическая>>}
Отличия научной рациональности от философской в том, что наука не включает в себя условия своего обоснования, но при этом обязательно непротиворечива. Философия, напротив, ялвяется полной системой в смысле самообоснования, но жертвует непротиворечивостью, тем не менее включая ее в качестве творческого источника жизни всей системы.
Согласно Степину, выделяются другие три типа: классический (элиминация субъекта), неклассический (связи между знаниями и характером средств и операций деятельности) и постнеклассический (рефлексия над деятельностью, ценностно-целевыми структурами).

Современная рациональность включает критико-рефлексивную установку по отношению к собственным предпосылкам, и предметом рационального сознания становится деятельность по выработке рационального знания на основе познавательных средств и предпосылок. Она возникает с пониманием невозможности универсального единого описания мира, в едином порядке обнаруживаются противоречия, приходит понимание, что наблюдаемое зависит от процесса наблюдения.

\section{Эволюция подходов к анализу науки в XX веке.}
Позитивистская традиция в философии науки. 
Концепции О. Конта, Л. Витгенштейна, К. Поппера, 
Т. Куна, П. Фейерабенда, М. Полани 
Впервые термин философия науки введен в 1840 г. Однако в качестве самостоятельной отрасли знания философия науки начинает развиваться в ХХ в. 
Философия науки — это то, что философы и ученые думают о науке, при этом думы о науке диаметрально противоположны. Одни признают позитивную ценность науки, а другие говорят о губительном последствии науки для развития человечества. 
Российский философ В. Степин считает, что предметом философии науки является изучение общих закономерностей научного познания, причем наука должна рассматриваться в исторически изменяющемся культурном и социальном контексте. 
Большой вклад в эволюцию подходов к изучению науки внёс позитивизм. Фундаментальной установкой позитивизма является утверждение о том, что истинное знание может быть получено как результат только научной деятельности. Что касается философии, то ей отказано в праве называться научным знанием. 
На базе развития позитивизма можно выделить три основных этапа развития взглядов на науку: 
1. позитивизм, 
2. неопозитивизм, 
3. постпозитивизМ 
Позитивизм как философия науки возникает в первой половине ХIХ в. Его основателем является О. Конт. Смысл подхода О. Конта к развитию науки заключается в учении о трёх стадиях развития знания. Человеческое познание в своем развитии проходит три стадии: 
1. теологи ческая, 
2. метафизическая, 
3. позитивная 
На теологической стадии человек объясняет явления природы действием сверхъестественных сил, Бога. 
На метафизической стадии человек объясняет явления природы действием абстрактных причин, сил. 
На позитивной стадии человек объясняет явления природы на основе данных науки. Здесь происходит объединение теории и практики. Человек довольствуется тем, что на основе наблюдений выделяет связи в природе, а на основе анализа постоянно повторяющихся связей формулирует законы. Третья стадия, таким образом, связана с господством научного подхода в объяснении мира. 
По Конту, наука перестаёт быть делом одиночек, она превращается в силу производства (наука определяет производство, выступает как производительная сила). Наука жестко связана с опытом и экспериментом. Только позитивное опытное знание способно дать что-то новое. Что касается философии, теологии, искусства, то они науками не являются. При этом О.Конт не отрицает важность философии или теологии, просто он полагает, что эти формы знания не относятся к наукам. 
Настоящая наука, по Конту, ничего не может сказать о причинах существования Космоса или человека. Она говорит о том, как устроен Космос или человек, но не о том, почему они существуют. Следовательно, наука должна отвечать на вопрос ((Как?», а не ((Почему?». 
К числу позитивных наук О. Конт относит математику, астрономию, физику, химию, биологию, социологию. Эти науки позитивные, так как основаны на опыте, они приносят практическую отдачу, их выводы можно использовать на практике. 
Неопозитивизм возникает в 30-е годы ХХ в. Неопозитивизм привлек внимание к необходимости философского анализа языка науки, к математической и формальной логике. 
Один из основателей неопозитивизма Л. Витгенштейн утверждает, что философия науки должна заниматься анализом языка науки. Наука нуждается в очищении своего языка. В науке очень много сведений, связанных с употреблением обыденного языка, а обыденный язык многозначен, что может исказить результаты научного исследования. Необходимо создавать особый язык науки. Язык науки очень тесно связан с опытом. Научные знания берутся из опыта, поэтому научные истины — это совпадение высказываний с опытом человечества или человека. В связи с этим 
Л. Витгенштейн выдвигает принцип верификации. Верификация — это опытная проверка научного знания на истинность. Следовательно, наука отличается от ненаучных знаний тем, что научные знания верифицируемы, то есть подтверждаются на опыте. Таким образом, Л. Витгенштейн и неопозитивизм в целом привлекли внимание к проблеме языка науки, к математической и формальной логике. 
другой представитель неопозитивизма К. Поппер показывает, что невозможно все содержание науки свести только к утверждениям, которые основаны на опыте, К. Поппер считает, что в науке прийти ко лжи гораздо легче, чем к истине. Первый признак ложности теории заключается в том, что все факты стремятся объяснить только из этой теории. 
В результате факты, которые противоречат господствующей теории, не учитываются, замалчиваются. Настоящая же наука какие-то явления и факты объясняет, а какие-то нет. Теория, а вместе с тем и наука является настоящей тогда, когда её принципы могут быть опровергнуты некоторыми фактами. В этом и состоит смысл идеи К. Поппера о фальсификации. Фальсификация — это возможность опровержения теории какими—то фактами. Принцип фальсификации отличает науку от ненаучных форм знания (теология, идеология, астрология и т.п.). 
Постпозитивизм развивается во второй половине ХХ в. В рамках постпозитивизма намечается очевидная установка на анализ социальных и культурных факторов в развитии науки. Крупнейшие представители постпозитивизма: Т. Кун, П. Фейерабенд, М. Полани. 
Т. Кун полагает, что наука непонятна вне своей истории и обращает внимание на необходимость исторического подхода к изучению наук. В этой связи он выдвигает концепцию парадигмы. Парадигма — это признаваемая на данном этапе большинством ученых господствующая научная идея (теория). Парадигма — это общепринятый в науке способ видения проблем. Термин «парадигма» совпадает с понятием «нормальная наука» (эволюционная фаза в развитии науки). На этой фазе большинство ученых разделяет господствующую теорию. Затем начинается революционная фаза развития науки, появляется много альтернативных теорий, которые противоречат старой парадигме. И, наконец, научное сообщество выбирает для себя новую парадигму. 
П. Фейерабенд выдвигает две взаимосвязанные идеи: а) идея размножения теорий, б) идея гносеологического анархизма. 
По его мнению, настоящий ученый стремится создать собственную научную теорию, которая противоречит другим существующим теориям. Поэтому наблюдается постоянное увеличение конкурирующих научных теорий. Несмотря на взаимное отрицание, борьба теорий полезна для развития науки. Но отсюда следует, что не может быть единого языка науки и единой парадигмы. А что же остается? Остается полный научный плюрализм, который проявляется в хаотичном нагромождении теорий. 
В результате в науке господствует гносеологический анархизм. Отрицаются любые догмы, признается ценность любой теории. В свободном обществе все научные традиции равны, поэтому гносеологический анархизм — это нормальное явление для современной науки.

П. Фейерабенд полагает, что в развитии науки значительную роль играет не только разум, но и нерациональные явления — страсти и чувства. В науке идет ожесточенная борьба не столько за истину, сколько за славу, власть, деньги. Новая теория сменяет старую теорию не потому, что она более истинна, а потому что поддерживается властью, пропагандой, средствами массовой информации. 
Британский философ М. Полани выступает как критик К. Поппера и выдвигает концепцию личностного знания в науке. В структуре познавательной деятельности он выделяет явные и неявные компоненты. Явное знание представлено в понятиях и теориях. Это знание можно приобрести посредством изучения научных трудов и учебников. Неявное знание — это личностное знание, которое вплетено в искусство экспериментирования и теоретические навыки ученых, в их пристрастия и убеждения. Неявное знание передается через непосредственные («из рук в руки») личные контакты ученых. Суть концепции личностного знания в науке описана в следующих словах: <<‘Я показал, что в каждом акте познания присутствует страстный вклад позна ю щей личности и что это не добавка, не свидетельство несовершенства, но насущный элемент знания)). 
М. Полани утверждает, что личность ученого играет огромную роль в процессе познания. Тот факт, что время одиночек в науке прошло, вовсе не отрицает личностного вклада в науку. да, современный ученый в основном работает в коллективе, современная наука — плод коллективного творчества. Но в то же время ничем нельзя заменить искусство и талант отдельного ученого. Необходимы личность, способности. Нельзя любого человека «с улицы» сделать ученым, необходимы особые качества интеллекта. Наука не продвигается вперед одними только методами. Важное значение имеют личность ученого, общение творческих личностей, непосредственное обучение ученика с учителем. 

\section{Логико-эпистемологический подход к осмыслению сущности науки.}

\section{Позитивистская традиция в философии науки.}
Позитивизм складывался под лозунгом борьбы с умозрительной философией, полагая метафизику излишней, за исключением ее роли как метанауки. По мнению основателя позитивизма О. Конта, преодоление умозрительного характера - неизбежное следствие взросления разума.

Человеческое мышление как организм проходит три стадии: теологическую или фиктивную (разум в примитивной потребности ищет причины всего, порождает причины, отсутствующие в природе), метафизическую или абстрактную (мышление объясняет то, чего никогда не существовало: бытие, сущность) и реальную или положительную (подчинение воображения наблюдению, критическая ревизия метафизических понятий, удаление бессмысленных вопросов).

Цель позитивного мышления - удовлетворение собственных потребностей.
Для Спенсера наука - расширенный и усложненный здравый смысл. Наиболее известная идея - концепция социального организма: общество - живой организм, страты выполняют различные функции.

Мах основывался на трех принципах: эпистемологический принцип <<экономии мышления>> (предвосхищение фактов в мысли, абстрактность, удаление объяснительной части и метафизических категорий, замена причинности функцией), гносеологический принцип нерасчлененности субъекта и объекта (тождество психического и физического, субъективного и объективного; цвета, тоны, давления - настоящие элементы мира, не вещи; научное знание - в описании функциональных связей с помощью численных величин) и принцип конвенциональной природы научной теории (математика - лишь конвенциональный инструмент)

\section{Расширение круга философских проблем в постпозитивистской философии науки.}
Ведущий методологический регулятив - релятивизм: научная теория есть результат переплетения различных процессовв обществе, организующих мышление ученого. Центральную роль играет полемика с кумулятивной моделью развития научного знания, исходившей из принципов: неподверженность сомнению однажды обоснованных теорий, отказ от пересмотра истории науки, неизменность накопленного запаса правильных знаний, неизбежность нахождения и безвозвратного удаления заблуждений, отграничение науки от ненаучных форм знания.

Из принципов сформировалась проблематика постпозитивизма: отбор и интерпретация фактов, соотношение теории и эмпирии (влияние теории на отбор фактов), проблема интернализма-экстернализма (влияние процессов в обществе на содержание науки).


\section{Философия науки в работах К.Поппера.}
Верификация не должна быть единственным методологическим ориентиром научного исследования, так как верифицировать можно что угодно. Индуктивный переход не гарантирует истинности. Решить обе проблемы можно, отказавшись от ллогического перехода от фактов к теории - факт внелогичен, логично мышление. Интерпретации и осмысления, возникшие независимо от фактов, пригодны для формулирования теории, то есть теория строится мышлением, черпающем материал в самом себе. Решение возможного роста числа вариантов теорий по одному набору фактов - выдвижение только фальсифицируемых теорий. То есть теория должна бытьлогически непротиворечивой и предполагать факты, которые опровергнут теорию в случае и обнаружения. Таким образом убирается переход от опыта к теории, из теории логически выводятся факты, способные ее фальсифицировать. Метод проб и ошибок - наиболее рациональная процедура. Эти взгляды строго отделяют научное знание от ненаучного.

\section{Теория научных революций Т.Куна.}
В <<Структуре научных революций>> расматривал вопрос выживаемости научных концепций, видя причиной смены не новые факты, а пересмотр отношения научного сообщества к аномалиям. Между сменяющимися парадигмами может не вестись логическая дискуссия, используются различные трактовки одних и тех же терминов, побеждает претендент, наиболее подходящий научному сообществу в данный момент. 
Следствие образования первой парадигмы: прекращение научных дискуссий по фундаментальным проблемам, исчезновение научных школ с иными взглядами, организуется научное образование соответственно парадигме, где принципы преподаются как подтвержденные и единственно возможные, облегчается труд ученого, так как он заранее знает, ка котобрать актуальные факты и как строить (уточнять) теорию, прекращаетсясоздание фундаментальных трудов и наука становится узкоспециализированной.
После смены парадигмы - новая картина мира признается единственно верной, преподавание наук переориентируется, переписываются учебники по истории науки.
Нормальная наука развивается кумулятивно через гипотетико-дедуктивный метод до следующего кризиса, при котором либо приспособится к аномалиям, либо переживет революцию.

Теория фрагмента реальности возникает после выделение одного объяснения случайно накопленных фактов по причинам социально-психологического характера, не более высокой степени мпирической подтвержденности. Парадигма формируется при охватывании всех известных областей реальности исходящими из одних фундаментальных принципов теорий. Она имеет два измерения: метафизическое (принципы организации мира) и социальное (то, что объединяет членов научного сообщества, эквивалентно ему). 

В <<Дополнении 1969 года>> выделяет четыре компонена: символические обобщения (облеченные в логическую форму выражения), метафизические части парадигмы (убеждение в истинности моделей), методологические ценности (критерий научности) и образцы (конкретные способы решения конкретных проблем)



\section{Синтез конвенционализма и фальсификационизма в концепции философии науки И.Лакатоса.}
Твердое ядро теории (ряд связанных высказываний и формул, внятно выражающих основную идею теории), отрицательная (запрет изменения ядра теории) и положительная (возможность изменений теории с сохранением ядра) эвристики формируют научно-исследовательскую программу.
Сменить теорию может независимая от первой теория-конкурент, которая имеет абсолютно отличное ядро, обладает отрицательной эвристикой (одинаковая для всех теорий), иной положительной эвристикой и более мощной эмпирической базой и эвристической силой.
Механизм фальсификации через другую теорию с иной концепцией реальности и подтвержлдаемой большим множеством фактов, в том числе фактами, подтверждавшими фальсифицируемую теорию.


\section{Идея <<исследовательских традиций>> Ларри Лаудана.}

\section{<<Методологический анархизм>> П.Фейерабенда.}
<<Против методологического принуждения>>,  <<Наука в свободном обществе>>

Пытался преодолеть объективность процесса формирования и смены научных теорий и ее негативные последствия. Избежать догматизма и застоя можно, устранив причину - отсутствие конкурирующих теорий. Принцип пролиферации - необходимость создания альтернативной теории с принципиально иным твердым ядром по Лакатосу через изобретение новой концептуальной системы, несовместимой с наиболее обоснованными результатами наблюдения и нарушающая самые правдоподобные теоретические принципы. не обращая внимания на трудности (принцип упорства). По принципу изменения значения (собственный язык в новой теории), принципу несоизмеримости (невозможность сравнительного анализа теорий) и принципу вседоступности допускаются любые концепции и теории и запрещается взаимная критика.

Наука должна быть отделена от государства, так как конкурентную борьбу выигрывает созвучная социально-политической обстановке теория.

\section{Социологический и культурологический подходы к исследованию развития науки.}

\section{Феноменологические и герменевтические аспекты анализа научного знания.}

\section{Наблюдение и эксперимент. Роль приборов в научном познании.}
Наблюдение - целенаправленное изучение и фиксирование данных об объекте в его естественном окружении, опирающиеся в основном на чувственные способности человека: ощущения, вопсприятия, представления. Структурные компоненты: наблюдатель, объект исследования, условия наблюдения, средства наблюдения.

Эксперимент - целенаправленное, четко выраженное активное изучение и фиксировмние данных об объекте в специально созданных и точно фиксированных и контролируемых условиях. Структурные компоненты: определенная пространственно-временная область, изуаемая система, протокол эксперимента, реакции системы.

Эксперимент оьладает рядом преимуществ: вопроизводимость, обнаружение характеристик, ненаблюдаемых в естественных условиях, возможность изолировать явление через варьирование условий, расширенные возможности использования приборов и автоматизации.

Эксперимент - связующее звено между эмпирическим и теоретическим этапами: он призван проверять определенные гипотезы, а его резултаты всегда интерпретируются с точки зрения теории. 

Роль приборов в усилении органов чувств, их дополнении новыми модальностями, повысить эффективность за счет ускорения, усиления и автоматизации некоторых мыслительных операций. Приборам свойственны погрешности, они способны вносить возмущения в наблюдаемый объект.

\section{Эмпирические факты и эмпирические зависимости. Процедуры формирования факта и его "теоретическая нагруженность".}

Теория погружена в интертеоретический контекст, специально сформулированные подтеории служат для сопоставления с опытом. Проверка теории проходит четыре ступени: метатеоретическую (непротиворечивость, недвусмысленность содержания, проверяемость), интертеоретическую (совместимость с другими теориями), философскую (оценка достоинств в свете философской концепции) и эмпирическую. 

\section{Эмпиризм и рационализм о соотношении теории и опыта.}
Эпмирический уровень научного знания включает наблюдение, эксперимент, группировку, классификацию, описание результатов наблюдений и экспериментов, моделирование. Теоретический уровень включает в себя выдвижение, построение и разработку научных гипотез и теорий, формулирование законов, выведение следствий, сопоставление различных гипотез и теорий, процедуры объяснения и предсказания.

Уровни различаются по предмету:эмпирическое исследование направлено на явления и зависимости, теоретическое исследование начелено на выявление сущностных связей. Отличаются также средства познания, понятия, используемые методы.

Дилемма рационализма и эмпиризма связана с убеждением, что философия Нового времени является по преимуществу эпистемологией, редуцирующей философские вопросы к поиску надежного основания познания. Рационализм решающую роль в познании приписывал разуму, эмпиризм - опыту. Эпистемологический характер теорий постулирует разрыв между субъектом познания и внешним миром.

\section{Логическое оформление теории. Логико-методологические принципы классификации научных понятий и терминов.}
По своей логической структуре классификация представляет собой операцию, основанную на делении понятий. Однако классификация отличается от деления понятий в двух отношениях:

1) если деление может производиться по любому возможному основанию, то классификация осуществляется по признаку, имеющему существенное значение для распределения исследуемых объектов. Большей частью она используется для систематизации накопленных знаний в разных областях науки, и поэтому носит более устойчивый характер, чем простое деление понятий;

2) при классификации распределение объектов производится по существенным признакам, в то время как деление можно провести по отличительным признакам. Очевидно, что такое отличие не является абсолютным хотя бы потому, что предпосылкой даже научной классификации служит первоначальное разграничение объектов и понятий по их отличительным, а не существенным признакам.

Таким образом, классификацией называется распределение объектов по тому или иному существенному свойству, в результате чего каждый из них попадает в точно указанный класс, подмножество или группу. Понятие классификации применимо, следовательно, не только к объемам понятий, но и к тем реальным предметам, которые подпадают под эти понятия. О классификации говорят также и тогда, когда расчленяют сложный предмет на его составные части. Такую классификацию называют мерологической.

В научном познании доминирующую роль играет таксономическая классификация, когда она проводится по типам, классам, родам и видам понятий, характеризующим соответствующие объекты реального мира. Наибольшее значение в науке имеет естественная классификация, основанная на распределении объектов и соответствующих им понятий на основе общности и существенности тех признаков, которые им присущи;

\section{<<Дилемма теоретика>> К.Гемпеля. Возможности устранения теоретических терминов (результаты Ф. Рамсея и У.Крейга)}
Наука не должна обращаться к гипотетическим сущностям, она заинтересована в установлении предсказательных и объяснительных связей между наблюдаемыми сущностями. Установление связи между наблюдаемыми объектами К. Гемпель называет систематизацией. Так, если теория устанавливает между наблюдаемыми объектами некоторую дедуктивно-номологическую связь, то можно говорить о дедуктивной систематизации, если же эта связь недедуктивная, вероятностная, то можно говорить об индуктивной систематизации. Термины не нужны, если не выполняют функции систематизации. В противном случае связи можно установить и напрямую, без дополнительных понятий.

Рамсэй-элиминация - переформулировка теории без использования неэмпирических понятий, когда теоретические термины заменяются логическими конструкциями. Его результаты показывают принципиальную утранимость терминов, но не снимает вопрос о необходимости терминов. Крейг получил более общий результат, однако он оказался громоздким и теория оказывалась непригодной.

\section{Дедуктивная и индуктивная систематизация научной теории.}
Индуктивный метод (индукция) характеризует путь познания от фиксирования опытных (эмпирических) данных и их анализа к их систематизации, обобщениям и делаемым на этой основе общим выводам. Данный метод заключается также в переходе от одних представлений о тех или иных явлениях и процессах к другим – более общим и чаще всего более глубоким. Основой функционирования индуктивного метода познания являются опытные данные. Так, основополагающие представления о современном капитализме, составляющие содержание соответствующих теорий, получены в результате научного обобщения исторического опыта развития капиталистического общества в последние 100 с лишним лет.

Однако индуктивные обобщения будут полностью безупречными лишь в том случае, если досконально изучены все научно установленные факты, на основе которых делаются эти обобщения. Это называется полной индукцией. Чаще всего сделать это очень трудно, а порой и невозможно.

Поэтому в познавательной деятельности, в том числе при исследовании различных явлений и процессов общественной жизни, чаще используется метод неполной индукции – изучение какой-то части явлений и распространение вывода на все явления данного класса. Обобщения, полученные на основе неполной индукции, в одних случаях могут носить вполне определенный и достоверный, в других – более вероятностный характер.

Достоверность индуктивных обобщений может быть проверена путем применения дедуктивного метода исследования, суть которого заключается в выведении из каких-то общих положений, считающихся достоверными, определенных следствий, часть которых может быть проверена опытным путем.

Если следствия, вытекающие из индуктивных обобщений, подтверждаются практическим опытом людей (экспериментом или реальными процессами общественной жизни), значит, эти обобщения можно считать достоверными, т.е. соответствующими действительности.

Следовательно, индукция и дедукция – это два противоположных и в то же время дополняющих друг друга метода научного исследования.

\section{Формализация и математизация теоретического знания.}
Формализация - методы познания, состоящии в отвлечении от содержания знания об объекте, понятий и других форм мышления, посредством которых выражено знание об объекте на естественном языке науки с дальнейшим исследованием объекта через изучение формы знания о нем, представленного в специальном, формализованном языке. Для возможности изучения формы знания, следующей за содержанием, требуется выявление и уточнение ее элементов и связей - эта уточненная форма и изучается при использовании метода формализации.

Структура: символизация (перевод на формализованный язык) - преобразование (операции по формальным правилам) - интерпретация (истолкование результатов на естественном языке) + практическая проверка

Стандарты: непротиворечивость представления, корректность (истинность результата), адекватность (выводимость истинного в формализованном представлении)

Достоинства: четкое выделение и представление предположений, возможность математической проверки и моделирования, выход в случае сложности формулировки на естественном языке.

\section{Гипотетико-дедуктивная схема развития научного познания.}
Основная идея гипотетико-дедктивной схемы - выдвижение определенных гипотез и последующая проверка непосредственно проверяемых следствий из них. Как правило имеют дело с совокупностью гипотез различного уровня, системы с дедуктивными связями между уровнями, где более низкий уровень соответствует следствиям из более высоких.

Выдвигаемая гипотеза основывается на данных опыта, так что описания опытных данных выводятся из нее, подвергается проверке. Опровержение не является основанием для отказа от гипотезы, но побуждением к дальнейшему исследованию.

\section{Критерии выбора теории.}
Теория должна быть принципиально проверяемой (возможность опровержения, отсутствие совместности с любым исходом опыта), максимально общной (из теории выводится широкая совокупность описаний и предсказаний), системной (нахождение в парадигме и целостность), обладать предсказательной силой и принципиальной простотой (минимум допущений, не связанных с предположениями)

\section{Философские основания науки. Роль философских идей и принципов в обосновании научного знания.}
Под основаниями науки понимают систему регулятивов, определяющих цель и способы получения научного знания, представление и понимание научной реальности, формы и степень обоснованности научного знания и его включения в культуру. Цель и способы определяются идеалами, нормами и критериями, обобщенное понимание и представление воплощается в научной картине мира, формы и степень обоснованности науки и ее включения в культурный контекст обеспечивают философские основания.

К логико-эпистемологическим номативам науки относят описание (выявление совокупности данных о свойствах и отношениях), объяснение (выработка понимания сущности возникновения, развития, функционирования), системность (анализ и соотнесение данных по типам и классам, введение новых типов и классов), доказательность и обоснованность (соответствие логике), эвристичность (способность предсказывать новые свойства и отношения, открытие новых уровней организации мира и новых типов объектов).

К социокультурным нормам относят прагматическую (способы применения знания), прогностическую (перспективы развития, футурологические модели, рекомендации на будущее), экспертную (анализ и оценка проектов и программ).

Научная картина мира как возникшая на основе обобщения и синтеза основных фактов, понятий и принципов система представлений о фундаментальных свойствах и закономерностях универсума,состоит из концептуального (философские принципы и категории, общенаучные положения и понятия) и чувственного (наглядные представления в виде моделей) компонентов. Современная научная картина мира состоит из естественнонаучного, технического и социально-гуманитарного блоков.

Можно выделить две подсистемы философских оснований научного знания: онтологическую (сеть категорий для задания понимания реальности) и эпистемологическую (категориальные схемы для описания процедур и результатов)

\section{Первичные теоретические модели и законы. Принцип ceteris paribus (<<при прочих равных условиях>>).}
Своеобразной клеточкой организации теоретических знаний на каждом из его подуровней является двухслойная конструкция — теоретическая модель и формулируемый относительно нее теоретический закон.

В качестве элементов теоретических моделей выступают абстрактные объекты (теоретические конструкты), которые находятся в строго определенных связях и отношениях друг с другом.

Теоретические законы непосредственно формулируются относительно абстрактных объектов теоретической модели. Они могут быть применены для описания реальных ситуаций опыта лишь в том случае, если модель обоснована в качестве выражения существенных связей действительности, проявляющихся в таких ситуациях

Согласно принципу ceteris paribus, факторы, не входящие в явном виде в формулировку результата, остаются неизменными, эксперимент проводится в обычных условиях.

\section{Проблема включения новых теоретических представлений в культуру.}
Новые теоретические представления в современной постиндустриальной культуре включаются в нее путем постоянной перестройки ее фундаментальных положений, влекущих перестройку социальной и духовной сфер, таким образом затрагиваются интересы каждого человека. В первую очередь новые представления касаются метафизических положений, и у них могут возникнуть сложности с традиционными представлениями, ставшими частью обыденного опыта. Любое новое теоретическое представление должно пройти сложную работу, иногда изменяющую его до неузнаваемости, несколько этапов цензуры (гласной и негласной, сознателной и бессознательной), в чем должно помогать образование, одной из важнейших функций которого становится производство социально востребованного типа личности.

\section{Научные революции как <<точки бифуркации>> в развитии знания. Нелинейность процесса роста знаний.}
Революции выступают как разрешение многогранного противоречия между старым и новым знанием, наличие двух фаз - эволюционной и революционной - выражение принципиальной нелинейности процесса роста знаний. Часто наука напоминает движение вспять, когда новые теории формулируются на основе ранееотброшенных идей. Революции вызываются ростом числа фактов, не поддающихся объяснению и связанной с этим необходимостью выработки новых теоретических представлений, кардинальной перестройкой картины мира и философским обоснованием новаций, включающим их в общекультруный фон.

Выделяют четыре типа научных революций: глобальную (вся наука), комплексную (ряд областей), частную (одна область) и научно-техническую (преобразование производительных сил).

\section{Историческое развитие институциональных форм научной деятельности.}

\section{Наука, экономика, власть. Проблемы организации, регулирования и контроля над научными исследованиями.}

\section{Главные характеристики современной  науки. Научный реализм и антиреализм.}
Фундамент эволюционно-синергетической научной картины мира составляют синергетика, системология и информационный подход, в рамках которого информация понимается как атрибут материи наряду с движением, пространством и временем. Развитие расматривается как универсальный и глобальный самодетерминированный нелинейный процесс самоорганизации нестационарныхоткрытых систем. Утверждается фундаментальная могласованность основных законов Вселенной с существованием в ней жизни и разума.

В общенаучной концепции универсального эволюционизма принцип развития воспроизводится на уровне оснований науки, служащей центром идейной кристаллизации новой научной картины мира. Элиминируется понятие изолированной системы и концепции абсолютного детерминизма, всякий локальный процесс эволюции объясняется толькокак необходимый момент единого универсального процесса развития Вселенной как целого. Произошло осознание целостности и системности Метагалактики, доступной научному познанию части мира. Развитие трактуется как нелинейный, вероятностный и необратимый процесс. Одно из центральных мест занимает антропный принцип, согласно которому человеческое бытие рассматривается как эндогенная форма бытия по отношению к миру и природе (<<Мир таков, потому что существует человек>>). Протекание эволюционных процессов представляется в форме самоорганизации сложных систем, обнаруживающих общие закономерности.

Научный реализм — течение в философии науки, согласно которому единственным надёжным средством достижения знания о мире является научное исследование, результат которого интерпретируется с помощью научных теорий. Теории научного реализма могут быть также вероятно истинными или приблизительно истинными или относительно истинными. Теории касаются наблюдаемых и ненаблюдаемых объектов, хотя и являются в сущности достоверными, однако могут быть в какой-то степени ложны. Исследуемые объекты независимы от нашего разума, а научные теории достоверны по отношению к внешнему (объективному) миру. Задача научного реализма — сформулировать истинные утверждения о реальности, что производится в сопровождении лучших научных теорий. Понятие реализма в научных терминах полезнее рассмотреть в трех измерениях:

Семантический реализм полагает, что теории являются истинными либо ложными описаниями реальности. Истинность и ложность зависит от того, существуют ли данные объекты и насколько верно они описаны теорией.
Метафизический реализм предполагает существование реальности независимо от нашего знания. Теория должна отвечать <<метафизическим представлениям>>. По словам Патнэма: <<метафизический реалист предлагает нам принять некоторую картину так, словно эта картина сама себя объясняет>>.
Эпистемологический реализм полагает, что доверие суждениям об истинности теории по преимуществу оправдано. Эпистемологический научный реалист считает, что наука сопровождается лучшими научными теориями, даже если они не могут быть доказаны с абсолютной уверенностью; можно предположить, что некоторые теории могут оказаться значительно ошибочными, тем не менее научный реалист уверен в том, что они в какой-то степени истинны.

Антиреализм утверждает, что наблюдение как основа научного знания является теоретически нагруженным, т. е. из него в принципе невозможно исключить некоторый элемент субъективности. Мы изменяем саму физическую реальность, когда пытаемся с ней взаимодействовать, чтобы изучить ее, а также теоретически нагружаем каждый акт наблюдения, интерпретируя его с позиции базового, принятого нами ранее знания.

\section{Научный натурализм и фундаментализм.}
Натурализм (фр. naturalisme; от лат. naturalis — природный, естественный) — философское направление, которое рассматривает природу как универсальный принцип объяснения всего сущего, причём часто открыто включает в понятие «природа» также дух и духовные творения (стоики, Эпикур, Дж. Бруно, Гёте, романтизм, биологическое мировоззрение XIX века, философия жизни).

По Канту, натурализм есть выведение всего происходящего из фактов природы. В этике — это требование жизни, согласующейся с законами природы, развитие естественных побуждений, а также философская попытка объяснить понятия морали чисто природными способностями, побуждениями, инстинктами, борьбой интересов (киники, стоики, Руссо, Конт, Маркс, Ницше).

ФУНДАМЕНТАЛИЗМ (в философии науки) – характеристика философско-методологических концепций, исходящих из существования некоего базисного, фундаментального слоя знания, обращение к которому позволяло бы гарантированно решить все задачи, связанные с уточнением познавательного содержания и обоснованием систематизируемого знания. Фундаменталистские концепции считают возможным сведение этого знания и соответственно выражающего его языка науки к такому фундаментальному слою или выведение из него. На позициях фундаментализма стояла классическая гносеология и методология науки Нового времени как в ее эмпирико-индуктивистском (Ф.Бэкон), так и в рационалистско-дедуктивистском варианте. Для первого варианта «твердым фундаментом» выступали эмпирические утверждения, а механизмом построения всего корпуса научного знания на этом фундаменте были методы научной индукции. Для второго варианта «твердым фундаментом» служили истины интеллектуальной интуиции (Декарт) или логически истинные аналитические утверждения (Лейбниц); соответственно механизмами выведения из них или сведения к ним выступали методы дедукции и логического анализа.

\section{<<Старая>> социология науки Роберта Мертона.}
Мертон формирует основы социологического анализа науки как особого социального института с присущими ему ценностно-нормативными регулятивами»

Цель (основная задача) науки, с точки зрения Мертона, заключается в постоянном росте массива удостоверенного научного знания. Для достижения этой цели необходимо следовать четырём основным императивам научного этоса: универсализм (внеличностный характер научного знания), коллективизм (сообщения об открытиях другим учёным свободно и без предпочтений), бескорыстие (выстраивание научной деятельности так, как будто кроме постижения истины нет никаких интересов) и организованный скептицизм (исключение некритического приятия результатов исследования).

По мнению Мертона, функциональный смысл указанных императивов ставит каждого учёного перед следующим набором альтернатив:

как можно быстрее передавать свои научные результаты коллегам, но не торопиться с публикациями
быть восприимчивым к новым идеям, но не поддаваться интеллектуальной моде
стремиться добывать знание, которое получит высокую оценку коллег, но работать, не обращая внимания на оценку результатов своих исследований
защищать новые идеи, но не поддерживать опрометчивые заключения
прилагать максимальные усилия, чтобы знать относящиеся к его области работы, но при этом помнить, что эрудиция иногда тормозит творчество
быть тщательным в формулировках и деталях, но не быть педантом
всегда помнить, что знание универсально, но не забывать, что всякое научное открытие делает честь нации, представителем которой оно совершено
воспитывать новое поколение учёных, но не отдавать преподаванию слишком много времени
учиться у крупного мастера и подражать ему, но не походить на него

\section{<<Сильная программа>> в эпистемологии науки.}
Сильная программа или строгая социология — концепция в социологии научного знания, развиваемая Дэвидом Блуром (David Bloor), Барри Барнсом (S. Barry Barnes), Гарри Коллинзом (Harry Collins), Дональдом Маккензи (Donald A. MacKenzie) и Джоном Генри (John Henry). Сильная программа критически относится к предшествующей социологии науки и исследует не только ошибочные или ложные научные теории и технологии, не нашедшие дальнейшего применения, но и успешные теории и технологии симметричным образом. Базовые предпосылки концепции — отказ наделять научное знание особыми свойствами и, как следствие, рассмотрение науки как специфической формы культуры

Дэвид Блур в книге «Знание и социальное представление» (Knowledge and Social Imagery (1976)) формулирует четыре необходимых компонента строгой программы:

Причинность (каузальность) — необходимо выявлять условия (психологические, социальные и культурные), которые устанавливают критерии научной объективности и достоверности определённого вида знания.

Беспристрастность — следует исследовать как признанные и принятые в конечном счёте в научном сообществе знания, так и те, которые были отвергнуты. Беспристрастность означает радикальное сомнение в собственных представлениях (стереотипах) об устройстве внешнего мира. В социологическом анализе следствием этой методологической установки является недоверие к представлениям учёных о своей деятельности.

Симметрия — подходы в объяснении как принятых, так и отвергнутых научных теорий должны быть идентичны. Сторонники концепции строгой теории отмечают склонность историков науки объяснять успех успешных теорий торжеством объективной истины или преимуществами техники и технических знаний, тогда как в объяснении появления и развития заблуждений и ложных представлений или не нашедших широкого применения технических достижений исследователи чаще опираются на социологические закономерности, рассматривают влияние политических или экономических предпосылок. 

Принцип симметрии как концепции строгой теории призван разрешить эту несбалансированность в объяснениях. Всегда можно выделить множество конкурирующих теорий и технологий в решении той или иной технической или научной задачи. Выбор и дальнейшее преобладание одной из них во многом объясняется социальными факторами. Иными словами, объяснение научной деятельности в терминах «доказательства», «истины» или «объективности» должно сочетаться с непосредственно социологическим объяснением, рассматривающим научные теории или их создателей в социальном контексте.
Рефлексивность — данные компоненты должны применяться и к анализу самой социологии, в том числе социологии научного знания.

\section{Глобальный эволюционизм и современная научная картина мира.}
Научная картина мира - синтетическое, систематизированное и целостное представление о природе на данном этапе научного познания. Ее эволюционность соответствует эволюционному характеру научного знания как такового, ее интегративность в том, что она является центром собирания, систематизации и согласования данных отдельных науксдля создания целостного образа мира.

Эволюция научной картины мира прошла этапы космоцентризма, теоцентризма, механистической картины, вероятностной картины и, наконец, информационной (информация - основная форма обобщения и передачи знания).

\section{Этика науки. Проблема ответственности учёных за их деятельность.}
Выделяются этически нейтральные и этически оцениваемые деяния и объекты.
Этика научной дискуссии состоит в контроле за отсутствием сознательных логических ошибок при аргументации, отказе от использования способов доказательства, при помощи которых можно доказать все что угодно (апеллирование к интуиции, ограниченности человеческого разума и т. п.), проведении четкой границы между научной позицией и личными качествами собеседника.

Этика публикаций заключается в публикации исключительно своих идей (обязательны ссылки на использованные работы других авторов), донесение в том числе отрицательных результатов, использовании для размещения специализированных научных журналов.

Мертон выделяет четыре регулятива научной деятельности: универсализм (предположение, что изучаемые явления везде протекают одинаково), коллективизм (научное знание должно быть достоянием всего научного сообщества), бескорыстность (главный стимул - поиск истины) и организованный скептицизм (перепроверка заимствованных данных, отказ от несостоятельных идей).

Социальная этика науки призвана ответить на вопрос, чем должен определяться научный прогресс: объективной логикой развития науки или социальной ответственностью ученого. В связи с этим возникают вопросы о том, кто несет ответственность за негативное использование результатов и о необходимости прекращать исследование, если последствия его использования наверняка окажутся деструктивными.

\section{Сциентизм и антисциентизм.}
Сциентизм базируется на абсолютизации рационально-теоретических компонентов знания. Он отстаивает позицию, согласно которой только научное знание является настоящим знанием, что методы и допущения (включая эпистемологические и метафизические учения), на которых основаны естественные науки, могут быть использованы в общественных и гуманитарных науках. Характерна преувеличенная вера в науку как средство получения знаний и решения стоящих перед человечеством проблем и приверженность натуралистической, материалистической, преимущественно механистической метафизике. Именно эта приверженность является ключевой особенностью сциентизма, и именно она служит главным объектом критики, поскольку сциентизм не просто отстаивает мощь науки, но выдвигает метафизические требования.

Антисциентизм (Гуссерль, Виндельбанд, Риккерт) опирается на ключевуюроль этических, правовых, культурных ценностей по отношению к идеалу научности. При любой рациональной доктрине не удастся освободиться от изначальной субъективности и влияния культурного контекста, от него отчужден комплекс научного знания, он не несет ответственности за применение своих открытий и не задумывается об их влиянии.



\chapter{Философия математики}
\section{Метафизические, семантические и эпистемологические проблемы математики}
\section{Математика как язык науки}
Современная наука в лице квантовой механики и чуть ранее теория относительности лишь прибавили абстрактности теоретическим объектам, вполне лишая их наглядности. Рассмотренное обстоятельство и выдвигает математику на роль языка науки. 
По мере роста абстрактности естествознания эта идея находила все более широкую реализацию, а на склоне XIX в. столетия уже вошла в практику научного исследования в качестве своего рода методологической максимы.
Ситуация порой складывается в науке так, что без применения соответствующего математического языка понять характер физического, химического и т.п. процесса невозможно. Не случайно признание П. Дирака, что каждый новый шаг в развитии физики требует все более высокой математики. 

Напрашивается вопрос, что же содержится в объективных явлениях такое математическое, благодаря чему они и поддаются описанию на языке математики, на языке количественных характеристик? Это однородные единицы вещества, распределяемые в пространстве и времени.  Отметим, что понимание однородности как условия применимости математического описания явлений пришло в науку довольно поздно. 

\section{Проблема «непостижимой» эффективности математики в естественных науках}
В статье Вигнер замечает, что математическая структура физической теории часто указывает путь к дальнейшим достижениям в этой теории и даже эмпирические предсказания.
Он верил, что математические концепции имеют приложения далеко вне контекста, где они были изначально разработаны. В пример он приводит пример закон всемирного тяготения, который, будучи изначально получен для моделирования свободного падения тел, далее успешно использовался на описания движения небесных тел, причем с неожиданной точностью. 
Другий пример - уравнения Максвелла, выведенные для моделирования элементарных электромагнитных феноменов, также описывают радиоволны. Он подводит итог, что огромная польза математики в естествознании граничит с мистикой и лишено рационального объяснения. По его мнению, это - дар, который мы не понимаем и который не заслужили.

Ричард Хэмминг представил четыре возможных объяснения:
1. Люди видят то, что ищут
2. Люди создают и выбирают математику, которая подходит в данном случае.
3. Математика объясняет только часть опыта
4. Математическое мышление - плод эволюции

\section{Конвенционализм в математике}
Конвенционализм (лат. conventio соглашение) — субъективно-идеалистическая философская концепция, согласно которой в основе математических и естественнонаучных теорий лежат произвольные соглашения (условности, определения, конвенции между учёными), выбор которых регулируется лишь соображениями удобства, целесообразности и т. п.

Основоположник конвенционализма — Пуанкаре.
Согласно Пуанкаре, основные положения любой научной теории не являются ни синтетическими истинами a priori, ни отражением реальности a posteriori. В связи с появлением неевклидовых геометрий он охарактеризовал системы аксиом различных математических теорий как соглашения, которые находятся вне поля истины или ложности. Предпочтение одной системы аксиом другой обусловлено принципом удобства. Единственное ограничение на их произвольный выбор состоит в требовании непротиворечивости.
При появлении более эффективных конвенций старые отвергаются.
След-но, все непротиворечивые научные (а также философские) теории в равной степени приемлемы и ни одна из них не может быть признана абсолютно истинной. Роль конвенций часто не осознается.

Следует различать два типа конвенций: индивидуальные (внутренние), которые можно рассматривать как скрытые определения, и социальные (внешние), которые имеют нормативный надындивидуальный характер.

Развитие математической логики в 1930-х привело к усилению позиций конвенционализма. С формально-логической точки зрения для мира объектов возможны отличные системы классификаций. Так, согласно "принципу терпимости" Карнапа, в основе данной научной теории может находиться любой "языковой каркас", то есть любая совокупность правил синтаксиса. "Принять мир вещей значит лишь принять определенную форму языка". "Языковые формы" следует использовать с учетом их полезности, при этом вопросы, которые касаются реальности системы объектов данной теории, по выражению Карнапа, оказываются сугубо внешними принятому "языковому каркасу".

Более крайней позицией явился "радикальный конвенционализм" Айдукевича, в соответствии с утверждением которого, отображение объектов в науке зависит от выбора понятийного аппарата (терминологии), причем этот выбор осуществляется свободно. К. Айдукевич предложил т. наз. радикальный конвенционализм, согласно которому в научной теории вообще нет неконвенциональных элементов.

К. Поппер считал, что конвенционален выбор базисных (опытных) предложений теории. Конвенционализм следует отличать от инструментализма: первый представляет из себя эпистемологически позитивную идею (теории являются конвенциями), а второй - эпистемологически негативную (теории не являются ни истинными ни ложными).

\section{Место философии в обосновании математики}
1. Платон: арифметика и геометрия – пути к созерцанию чистого бытия.

2. Аристотель: одна из частных наук. О том, что не существует само по себе и неподвижно. Математик выделяет себе отдельную область существующего.

3. Новое время: математика становится образцом для философии – по форме, хотя математика не выступает обоснованием философии, напротив, математика должна быть обоснована в метафизике. Классический пример «Этика» Бенедикта Спинозы Ethica ordine geometrico demonstrata (ок. 1675). (Определения, аксиомы, теоремы, схолии). Считается возможным доказательство философских положений, по типу математического (геометрического) доказательства.

4. Кант: Философия и математика – два вида априорного познания. Но математика соотносится со сферой чувственного, но не зависящего от материи опыта (понятие чистой чувственности).
«Принято утверждать, что математика и философия различаются друг от друга по объекту, поскольку первая трактует о количестве, а вторая о качестве. Все это неверно. Различие этих наук не может корениться в объекте, ибо философия касается всего, а следовательно, и количества (quanta), как отчасти и математика, поскольку все имеет величину. Специфическое различие между двумя этими науками составляют лишь различия в способе рационального познания, или применения разума в математике и философии. А именно философия есть рациональное знание из одних только понятий, математика же, напротив, рациональное познание посредством конструирования понятий». Кант И Логика. Пособие к лекциям 1800 года // Кант И. Трактаты. СПб, 1996, стр. 436 – 437.
Но Кант спровоцировал и попытки поставить математику на место первой науки. Философия как метафизика – невозможна. Математика – наука обо всем, поскольку все выразимо количественно. Математическое естествознание.

5. Гегель. Философия как наука не может пользоваться математическим аппаратом, должна разрабатывать свой понятийный аппарат.
«Очевидность этого несовершенного познавания, которым математика гордится и кичится перед философией, покоится лишь на бедности ее цели и несовершенстве ее материала, а потому это такая очевидность, которую философия должна отвергать. – Цель математики или ее понятие есть величина. А это есть как раз несущественное, лишенное понятия отношение» Гегель Г.В.Ф. Феноменология духа, СПб, 1994, стр. 23.

6. В современности множество попыток построения математических философий – математические истины кладутся в основании философских, либо заменяют философские. Б. Рассел – математическая философия. А. Бадью: онтология = теория множеств (ZF).

\section{Проблема недоопределенности математической теории. Существование неизоморфных моделей}
\section{Теорема Левенгейма-Сколема, «парадокс» Сколема}
Теорема Лёвенгейма-Сколема — теорема теории моделей о том, что если множество предложений в счётном языке первого порядка имеет бесконечную модель, то оно имеет счётную модель. Эквивалентная формулировка: каждая бесконечная модель счётной сигнатуры имеет счётную элементарную подмодель.
Это утверждение впервые сформулировано в работе Леопольда Лёвенгейма 1915 года, доказано Туральфом Сколемом в 1920 году.

Пусть структура ${\mathfrak  N}$ является моделью множества формул счётного языка $ {\mathcal  L}$. Построим цепочку подструктур ${\mathfrak  {M}}_{n}$, $1\leqslant n<\infty $. Для каждой формулы $\varphi (x)\in {\mathcal  {L}}$ такой, что ${\mathfrak  {N}}\models \exists x\,\varphi (x)$, обозначим через $b_{{\varphi (x)}}$ произвольный элемент модели, для которого ${\mathfrak  {N}}\models \varphi (b_{\varphi })$. Пусть ${\mathfrak  {M}}_{1}$ подструктура ${\mathfrak  {N}}$, сгенерированная множеством $\{b_{{\varphi (x)}}\mid {\mathfrak  {N}}\models \exists x\,\varphi (x)\}$.
Индуктивно определим ${\displaystyle {\mathfrak {M}}_{n+1}} {\mathfrak  {M}}_{{n+1}}$ как подструктуру, сгенерированную множеством

$\{b_{{\varphi (x,\;{\bar  {a}})}}\mid {\mathfrak  {N}}\models \exists x\,\varphi (x,\;{\bar  {a}}),\;{\bar  {a}}\in {\mathfrak  {M}}_{n}\}$.
Так как количество формул счётно, каждая из подструктур ${\mathfrak  {M}}_{n}$ счётна. Заметим также, что их объединение удовлетворяет критерию Тарского — Вота, и следовательно является элементарной подструктурой ${\mathfrak  {N}}$, что и завершает доказательство.

Парадокс Скулема — противоречивое рассуждение, описанное впервые норвежским математиком Туральфом Скулемом, связанное с использованием теоремы Лёвенгейма — Скулема для аксиоматической теории множеств.

В отличие от парадокса Рассела, парадокса Кантора, парадокса Бурали-Форти, где при помощи логически верных выводов обнаруживается противоречие, «замаскированное» в исходных посылках, «противоречие» парадокса Скулема возникает от ошибки в рассуждениях, и аккуратное рассмотрение вопроса показывает, что это лишь мнимый парадокс. Тем не менее, рассмотрение парадокса Скулема имеет большую дидактическую ценность.

Формулировка: Если система аксиом любой аксиоматической теории множеств непротиворечива, то она, в силу теорем Гёделя и Лёвенгейма — Скулема, имеет модель и, более того, эта модель может быть построена на натуральных числах. То есть, всего лишь счётное множество объектов $M$ (каждый из которых будет соответствовать уникальному множеству) требуется для того, чтобы подобрать значение предиката $x\in y$ для каждой пары объектов, полностью удовлетворяющее аксиомам этой теории (например, ZF или ZFC — в предположении их непротиворечивости). В такой ситуации для каждого объекта модели $y$ лишь конечное или счётное количество объектов (больше просто нет в предметной области) могут входить в отношение $\ldots \in y$. Фиксируем такую модель ${\mathfrak {M}}$ со счётным $M$ в качестве предметной области.

В силу теорем ZF, вне зависимости от принятой модели в ZF выводимо, например, существование терма ${\mathcal {P}}(\omega )$, мощность которого несчётна. Но в счётной модели любое множество вынужденно не более, чем счётно — противоречие?

Разрешение: Проведём рассуждение аккуратно. Факт $\mathrm {ZF} \vdash \exists x(x={\mathcal {P}}(\omega ))$ означает, что существует такой объект $c\in M$, что формула первого порядка, соответствующая выражению $x={\mathcal {P}}(\omega )$, истинна в модели ${\mathfrak {M}}$ на оценке, при которой индивидной переменной $x$ поставлен в соответствие объект $c$. Теорема Кантора утверждает, что $x$ — несчётно, что по определению значит

$\mathrm {ZF} \vdash \neg \exists f$ (f — биекция между ${\mathcal {P}}(\omega )$ и $\omega )\land \exists f(f$ — биекция между $\omega$ и $\omega \cup {\omega })$
где « $f$ — биекция между $A$ и $B$» означает $\forall x\forall y(\langle x,\;y\rangle \in f\Leftrightarrow (x\in A\land y\in B))$, где $\langle x,\;y\rangle$ — любое кодирование упорядоченных пар, например, $\langle x,\;y\rangle =\{x,\;y,\;\{x\}\}$.

Но это значит лишь то, что среди элементов $M$ нет такого $f$, что в модели ${\mathfrak {M}}$ оно удовлетворяло бы свойствам биекции между ${\mathcal {P}}(\omega )$ и $\omega$. При этом не важно, что в отношение принадлежности с объектом из $M$, соответствующим терму ${\mathcal {P}}(\omega )$ может входить не более чем счётное число объектов из $M$ — важно то, что среди объектов $M$ не существует $f$, осуществляющего необходимую биекцию.

Рассуждение «если модель счётна, то в отношение $\in$  с любым объектом может входить не более чем счётное число объектов» есть рассуждение внешнее по отношению к изучаемой аксиоматической теории и никакой формуле в этой теории не соответствует. С внешней точки зрения на теорию ZF «множество всех множеств» (второй раз слово «множество» здесь обозначает лишь некоторый объект предметной области ZF) может существовать и даже быть счётным, что никак не связано (и потому не может противоречить) с выводимыми в ZF формулами.


\section{Независимость континуум-гипотезы и ее отнологические последствия}
Континуум-гипотеза (проблема континуума, первая проблема Гильберта) — выдвинутое в 1877 году Георгом Кантором предположение о том, что любое бесконечное подмножество континуума является либо счётным, либо континуальным. Другими словами, гипотеза предполагает, что мощность континуума — наименьшая, превосходящая мощность счётного множества, и «промежуточных» мощностей между счетным множеством и континуумом нет, в частности, это предположение означает, что для любого бесконечного множества действительных чисел всегда можно установить взаимно-однозначное соответствие либо между элементами этого множества и множеством целых чисел, либо между элементами этого множества и множеством всех действительных чисел.

Первые попытки доказательства этого утверждения средствами наивной теории множеств не увенчались успехом, в дальнейшем показана невозможность доказать или опровергнуть гипотезу в аксиоматике Цермело — Френкеля (как с аксиомой выбора, так и без неё).

Континуум-гипотеза однозначно доказывается в системе Цермело — Френкеля с аксиомой детерминированности (ZF+AD).

Континуум-гипотеза стала первой из двадцати трёх математических проблем, о которых Гильберт доложил на II Международном Конгрессе математиков в Париже в 1900 году. Поэтому континуум-гипотеза известна также как первая проблема Гильберта.

В 1940 году Курт Гёдель доказал в расширенной теории, полученной присоединением к системе аксиом Цермело — Френкеля (ZFC) аксиомы о непротиворечивости ZFC, что отрицание континуум-гипотезы недоказуемо в ZFC; а в 1963 году американский математик Пол Коэн доказал в той же теории, что континуум-гипотеза недоказуема в ZFC. Таким образом, континуум-гипотеза не зависит от аксиом ZFC. Вопрос о независимости континуум-гипотезы от аксиом использовавшейся Гёделем и Коэном расширенной теории остается открытым.

В предположении отрицания континуум-гипотезы $\mathrm {ZFC+\neg CH}$  имеет смысл задавать вопрос: для каких ординалов $\alpha$  может выполняться равенство $\mathfrak {c}=\aleph _{\alpha }$? Ответ на этот вопрос даёт доказанная в $1970$ году теорема Истона (англ.).

Теория Кантора, решив огромную часть проблем, связанных с пониманием бесконечности, вместе с тем поставила проблемы, которые до сих пор находятся в центре внимания философов и математиков. Такая ситуация довольно типична для науки, поскольку каждая новая теория ставит новые нерешенные проблемы.

Философская проблема состоит в том, что новые проблемы могут оказаться либо псевдопроблемами, либо неразрешимыми. Такая ситуация может возникнуть из-за того, что выразительные средства новой теории слишком богаты для постановки вопросов, но недостаточны для их разрешения. Относится ли все вышесказанное к континуум-гипотезе, трудно сказать, поскольку она имеет громкую историю и в определенном смысле взывает к параллели с неевклидовой геометрией.
Результат независимости континуум-гипотезы ставит вопрос о том, имеет ли это утверждение истинностное значение вообще. В пользу таких сомнений может быть приведено два соображения — одно философское, другое — математическое. Философское было высказано в начале этого раздела, когда говорилось, что слишком богатые выразительные средства, созданные для решения одной задачи, могут привести к формулировке вопросов, на которые нет ответа. Такой подход к пониманию природы континуум-гипотезы во многом перекликается с мнением Геделя: <<Однако эта негативная позиция... никоим образом не является результатом тщательного рассмотрения оснований математики, и является результатом лишь определенной философской позиции в отношении природы математики>>


\section{Квантовый бит и квантовые вычисления}

Квантовый компьютер — вычислительное устройство, которое использует явления квантовой суперпозиции и квантовой запутанности для передачи и обработки данных. Квантовый компьютер (в отличие от обычного) оперирует не битами (способными принимать значение либо 0, либо 1), а кубитами, имеющими значения одновременно и 0, и 1. В результате можно обрабатывать все возможные состояния одновременно, достигая гигантского превосходства над обычными компьютерами в ряде алгоритмов

Идея квантовых вычислений состоит в том, что квантовая система из L двухуровневых квантовых элементов (квантовых битов, кубитов) имеет 2L линейно независимых состояний, а значит, вследствие принципа квантовой суперпозиции, пространство состояний такого квантового регистра является 2L-мерным гильбертовым пространством. Операция в квантовых вычислениях соответствует повороту вектора состояния регистра в этом пространстве. Таким образом, квантовое вычислительное устройство размером L кубит фактически задействует одновременно 2L классических состояний.

Физическими системами, реализующими кубиты, могут быть любые объекты, имеющие два квантовых состояния: поляризационные состояния фотонов, электронные состояния изолированных атомов или ионов, спиновые состояния ядер атомов, и т. д.

Упрощённая схема вычисления на квантовом компьютере выглядит так: берётся система кубитов, на которой записывается начальное состояние. Затем состояние системы или её подсистем изменяется посредством унитарных преобразований, выполняющих те или иные логические операции. В конце измеряется значение, и это результат работы компьютера. Роль проводов классического компьютера играют кубиты, а роль логических блоков классического компьютера играют унитарные преобразования. Такая концепция квантового процессора и квантовых логических вентилей была предложена в 1989 году Дэвидом Дойчем. Также Дэвид Дойч в 1995 году нашёл универсальный логический блок, с помощью которого можно выполнять любые квантовые вычисления.

Оказывается, что для построения любого вычисления достаточно двух базовых операций. Квантовая система даёт результат, только с некоторой вероятностью являющийся правильным. Но за счёт небольшого увеличения операций в алгоритме можно сколь угодно приблизить вероятность получения правильного результата к единице.

С помощью базовых квантовых операций можно симулировать работу обычных логических элементов, из которых сделаны обычные компьютеры. Поэтому любую задачу, которая решена сейчас, любой квантовый компьютер решит, и почти за такое же время.

Большая часть современных ЭВМ работают по такой же схеме: n бит памяти хранят состояние и каждый такт времени изменяются процессором. В квантовом случае система из n кубитов находится в состоянии, являющемся суперпозицией всех базовых состояний, поэтому изменение системы касается всех 2n базовых состояний одновременно. Теоретически новая схема может работать намного (в экспоненциальное число раз) быстрее классической. Практически (квантовый) алгоритм Гровера поиска в базе данных показывает квадратичный прирост мощности против классических алгоритмов.


\section{Теоремы Геделя о неполноте и их возможные философские интерпретации}

Результаты Курта Гёделя в свете проблем информатики: он доказал 2 теоремы, которые разрушили программу финитизма в основании математики, пропагандируемую Гильбертом. Сие есть одно из самых фундаментальных открытий науки 20-го столетия. Некоторые авторы - результаты свидетельствуют о невозможности ИИ, и об ограниченности формальных методов науки.

Теоремы Гёделя:

Первая теорема Гёделя о неполноте

Во всякой достаточно богатой непротиворечивой теории первого порядка (в частности, во всякой непротиворечивой теории, включающей формальную арифметику), существует такая замкнутая формула F, что ни F, ни !F не являются выводимыми в этой теории.

Иначе говоря, в любой достаточно сложной непротиворечивой теории существует утверждение, которое средствами самой теории невозможно ни доказать, ни опровергнуть. Например, такое утверждение можно добавить к системе аксиом, оставив её непротиворечивой. При этом, для новой теории (с увеличенным количеством аксиом) также будет существовать недоказуемое и неопровержимое утверждение.

Вторая теорема Гёделя о неполноте

Во всякой достаточно богатой непротиворечивой теории первого порядка (в частности, во всякой непротиворечивой теории, включающей формальную арифметику), формула, утверждающая непротиворечивость этой теории, не является выводимой в ней.

Иными словами, непротиворечивость достаточно богатой теории не может быть доказана средствами этой теории. Однако вполне может оказаться, что непротиворечивость одной конкретной теории может быть установлена средствами другой, более мощной формальной теории. Но тогда встаёт вопрос о непротиворечивости этой второй теории, и т.д.

Караваев — теорему Гёделя трактуют как результат, ограничивающий возможости человеческого познания вообще, но это не так — теорема Гёделя свидетельствует об открытости мира для нашего познания — если мы возьмём недоказуемую формулу и прибавим её к набору аксиом, получим новую теорию.

Цилищев — математик никогда не имеет дела в своей реальной работе с бесконечным количеством утверждений, поэтому беспокоиться о противоречивости некоторой системы в общем-то не требуется. Ему не надо думать о том, что где-то на бесконечном наборе высказываний ему может встретиться противоречие, ему надо, чтобы тот фрагмент знания, с которым он работает, был непротиворечив. Так же и компьютер — он не будет порождать бесконечное.

Каждая формализация сама порождает прецедент, входящий в идеальное множество, но не подходящий под нее саму. Более того, таким свойством обладает и каждая вычислимая последовательность формализаций. Любая непротиворечивая теория сама помогает построить пример неразрешимого в ней истинного утверждения. (c) Непейвода


\section{Д. Лукас и Д. Хофштадтер о возможностях <<мыслящих>> машин и человеческого интеллекта}

Minds, Machines and Gödel is J. R. Lucas's 1959 philosophical paper in which he argues that a human mathematician cannot be accurately represented by an algorithmic automaton. Appealing to Gödel's incompleteness theorem, he argues that for any such automaton, there would be some mathematical formula which it could not prove, but which the human mathematician could both see, and show, to be true.

Lucas maintains the falseness of Mechanism - the attempt to explain minds as machines - by means of Incompleteness Theorem of Gödel. Gödel’s theorem shows that in any system consistent and adequate for simple arithmetic there are formulae which cannot be proved in the system but that human minds can recognize as true; Lucas points out in his turn that Gödel’s theorem applies to machines because a machine is the concrete instantiation of a formal system: therefore, for every machine consistent and able of doing simple arithmetic, there is a formula that it can’t produce as true but that we can see to be true, and so human minds and machines have to be different. Lucas considers as well in this article some possible objections to his argument: for any Gödelian formula we could, for instance, construct a machine able to produce it (indeed the procedure whereby a Gödelian formula is constructed is a standard procedure) or we could put the Gödelian formulae that we had proved as axioms of a further machine. However - as Lucas underlines - for every of such machines we could again formulate another Gödelian formula, the Gödelian formula of these machines, that they are not able to proof but that we can recognize as true. More general arguments, such as the possibility to escape Gödelian argument by suggesting that Gödel’s theorem applies to consistent systems while we could be inconsistent ones, are moreover refuted by Lucas by maintaining that our inconsistency corresponds to occasional malfunctioning of a machine and not to his normal inconsistency; indeed, a inconsistent machine is characterized by producing any statement, on the contrary human being are selective and not disposed to assert anything.

Douglas Hofstadter tells us in Godel, Escher, Bach: An Eternal Golden Braid (1979) that the rapid development of computer technology in the past two decades has brought about a new kind of perspective on just what thought is--its powers and weaknesses, as well as its idiosyncracies. This has been made possible through computer experimentation with what he calls "alien, yet hand-tailored forms of thought--or approximations of thought." The results of this experimentation are reflected in the theory and designs developed in the artificial intelligence (AI) field, and many of these developments are considered at some length in Godel, Escher, Bach.

\section{Математический реализм, его разновидности}
РЕАЛИЗМ МАТЕМАТИЧЕСКИЙ - истолкование математических объектов как имеющих реальную основу до образования математических теорий с принятыми в них понятиями. В самом широком понимании этого термина под него подпадают и традиционный эмпиризм, истолковывающий математические понятия как отражение некоторых аспектов опыта, и операционализм, связывающий их с операциями деятельности, и, наконец, платонизм, связывающий существование математических объектов с миром внечувственных реальностей. В общем плане реализм противостоит конвенционализму, априоризму и фикционализму, которые склонны рассматривать математические понятия исключительно как продукт мыслительной деятельности субъекта. Реализм противостоит также номинализму в том смысле, что он не ограничивает употребление математических абстракций областью единичных объектов. В этом смысле Р.м. имеет прямую связь со средневековым реализмом, который приписывал особое бытие общим категориям, независимое от существования индивидуальных объектов. В современной философии математики слово «реализм» употребляется часто как синоним платонизма, хотя ясно, что существуют основания для различения этих понятий.

Как уже было сказано, существует множество разновидностей реализма. Так, если утверждается независимое от сознания существование таких абстрактных объектов как множества, числа, общие понятия, то тогда мы имеем дело с понятийным реализмом или платонизмом; если речь идет о том, что основные научные понятия представляют действительно существующие объекты и процессы, то имеет место научный реализм; если же принимается объективное существование моральных норм и ценностей, то налицо – этический реализм. «Наивный» реализм рассматривает, в качестве реально существующих, совокупность обычных «макроскопических» предметов окружающего нас мира. Теоретико-познавательный реализм добавляет к вышеупомянутым онтологическим тезисам еще один, так называемый эпистемологический тезис:

(3) Реально существующие и образующие «внешний мир»объекты могут быть предметом человеческого опыта и познания.

Можно отметить, что хотя эпистемологический тезис и предполагает принятие обоих онтологических тезисов, вполне возможно принять тезисы (1) и (2) без того, чтобы разделять тезис (3). Примером здесь может служить трансцендентальный идеализм Канта, с его признанием объективного существования «вещи в себе», которая, тем не менее, не может быть дана нам в качестве предмета познания.

Современный реализм, в отличие от таких своих предшественников, как Дж.С. Милль, не отождествляет математическое познание с естественнонаучным. Однако новые тенденции в современной философии математики подчеркивают поворот математики к практике, который выражается в использовании компьютеров в доказательствах, в числовом экспериментировании, в признании различных версий доказательств и т.п. В связи с этим происходит постепенный отказ от прежних чрезмерно абстрактных концепций математики и критериев строгости ее доказательств, значительную роль в ней начинают играть конструктивные и вычислительные методы. Все это обусловило возрождение интереса к эпистемологическим проблемам математики. Математическое познание стало рассматриваться как специфический, но связанный с общенаучным, процесс развития мышления, в ходе которого допускаются и преодолеваются ошибки, а полученные результаты находят применение в естественных, технических и социальных науках. 

\section{Логицизм Фреге и Рассела. Неологицизм}
Лидеры логицизма (Г. Фреге, Б. Рассел и др.) видели основания математики в логике. Вообще, возможны четыре типа соотношения математики и логики: 1) Математика - часть логики; 2) Логика - часть математики; 3) Математика и логика имеют некую общую пранауку, из которой они произошли; 4) Математика и логика - совершенно разные дисциплины, не имеющие общих корней.

Первые идеи логицизма сформулированы в конце XIX в. в работах немецкого математика и логика Г.Фреге и развиты в самом начале XX в. Б. Расселом.
В основе логицизма лежит убеждение, что математика - своего рода надстройка к фундаменту, заложенному логикой и что математические объекты покоятся на логических основаниях. Иначе говоря, логицизм вообще полагает математику лишь частью, отраслью логики.

Будучи лишь частью логики, математика, по мнению логицистов, не должна заимствовать ни у созерцания, ни у опыта никакого обоснования. Все специальные математические термины могут быть представлены кратким перечнем основных понятий, которые принадлежат словарю чистой логики. Доказательство же математических теорем не требует иных аксиом, кроме логических, и правил вывода, помимо тех, что использует логика.

\section{Математический формализм. Программа Гильберта}
Формализм — один из подходов к философии математики, пытающийся свести проблему оснований математики к изучению формальных систем. Наряду с логицизмом и интуиционизмом считался в XX веке одним из направлений фундаментализма в философии математики.

Формализм возник в начале XX века в математической школе Гильберта в рамках попытки свести в единую систему строгие обоснования различных областей математики. Развивался сотрудниками (учениками) Гильберта Аккерманом, П. Бернайсом, фон Нейманом.

В отличие от логицизма, формализм не претендовал на построение единой для всей математики формальной теории, наподобие теории множеств или теории типов. В отличие от интуиционизма, формализм не отказывался от построения теорий с «сомнительными» с точки зрения интуиции основаниями, лишь бы в них правила вывода теорем были строго обоснованы. Формалисты полагали, что математика должна изучать как можно больше формальных систем.

Формализм возник в начале XX века в математической школе Гильберта в рамках попытки свести в единую систему строгие обоснования различных областей математики. Развивался сотрудниками (учениками) Гильберта Аккерманом, П. Бернайсом, фон Нейманом.

В отличие от логицизма, формализм не претендовал на построение единой для всей математики формальной теории, наподобие теории множеств или теории типов. В отличие от интуиционизма, формализм не отказывался от построения теорий с «сомнительными» с точки зрения интуиции основаниями, лишь бы в них правила вывода теорем были строго обоснованы. Формалисты полагали, что математика должна изучать как можно больше формальных систем.

Формально-аксиоматические теории, построенные на основе классической логики, имеет смысл рассматривать лишь при отсутствии в них противоречий, поскольку в противном случае «доказанным» оказывается любое суждение теории. Если в такой формальной системе удаётся доказать логическую ложь, то она находится противоречивой и «выбраковывается», что обесценивает любые доказанные в рамках данной системы теоремы. Разумеется, математиков волновал вопрос, можно ли каким-то образом доказать непротиворечивость теории. К досаде формалистов, было показано, что вопрос о противоречивости теории не имеет адекватного разрешения внутри любой из употребительных в математике формальных систем.

Ничто не мешает изучать одну формальную теорию при помощи другой; такой подход называется метаматематическим. Однако, он вынуждает использовать для построения метатеорий наиболее надёжные основания, каковыми формалисты рассматривали, опять-таки, классическую логику и формальную арифметику.

\section{Современный формализм Хаскеля Карри}
The focus of Curry's work were attempts to show that combinatory logic could provide a foundation for mathematics. Towards the end of 1933, he learned of the Kleene–Rosser paradox from correspondence with John Rosser. The paradox, developed by Rosser and Stephen Kleene had proved the inconsistency of a number of related formal systems including one proposed by Alonzo Church (a system which had the lambda calculus as a consistent subsystem) and Curry's own system.[2] However, unlike Church, Kleene, and Rosser, Curry did not give up on the foundational approach, saying that he did not want to "run away from paradoxes."[3]

By working in the area of Combinatory Logic for his entire career, Curry essentially became the founder and biggest name in the field. Combinatory logic is the foundation for one style of functional programming language. The power and scope of combinatory logic are quite similar to that of the lambda calculus of Church, and the latter formalism has tended to predominate in recent decades.

In 1947 Curry also described one of the first high-level programming languages and provided the first description of a procedure to convert a general arithmetic expression into a code for one-address computer.[4]

He taught at Harvard, Princeton, and from 1929 to 1966, at the Pennsylvania State University. In 1942, he published Curry's paradox. In 1966 he became professor of logic and its history and philosophy of exact sciences at the Universiteit van Amsterdam, the successor of Evert Willem Beth.[5]

Curry also wrote and taught mathematical logic more generally; his teaching in this area culminated in his 1963 Foundations of Mathematical Logic. His preferred philosophy of mathematics was formalism (cf. his 1951 book), following his mentor Hilbert, but his writings betray substantial philosophical curiosity and a very open mind about intuitionistic logic.

\section{Интуиционизм и интуиционистская логика. Алгебры Гейтинга}
Иммануил Кант и Леопольд Кронекер – предвестники математического интуиционизма, хотя наиболее известными его представителями являются Ян Брауэр и его ученик Аренд Гейтинг. Гейтинг утверждает:
Математик-интуиционист предлагает заниматься математикой как естественной функцией интеллекта, как свободным, важным занятием для развития мышления. Для него математика – это продукт человеческого разума.

А Брауэр говорит:
Математические категории не существуют в нашей концепции представлений о природе больше, чем сама природа.

\section{Возможные миры: семантика С. Крипке}
Семантика Крипке является распространенной семантикой для неклассических логик, таких как интуиционистская логика и модальная логика. Она была создана Солом Крипке в конце 1950-х — начале 1960-х годов. Это было большим достижением для развития теории моделей для неклассических логик.

Это концепция в логике и аналитической философии, придуманная для работы с модальными высказываниями. Модальные высказывания содержат слова, выражающие модальность (модальные операторы) - например, модальности истины (возможно, невозможно, необходимо...), времени (когда-нибудь, когда-то, всегда...) и т.д.

В семантике существует различие между экстенсиональностью и интенсиональностью. Экстенсионал имени это его референт (объект, на который оно ссылается), экстенсионал предиката - множество вещей, к которым он применяется, а экстенсионал высказывания - его истинностное значение. Интенсионал же можно назвать "смыслом" выражения, указывающим на его экстенсионал.

Логика является экстенсиональной, если истинность каждого ее высказывания определяется только его формой и экстенсионалами составляющих компонентов - высказываний, предикатов и имен. Поэтому в экстенсиональной логике действует набор валидных принципов субституции - если два выражения имеют один и тот же экстенсионал, то они взаимозаменимы в любом высказывании, не меняя истинностного значения высказывания. Известный пример Фреге: древние греки называли планету Венера "Утренней звездой" и "Вечерней звездой", считая ее двумя разными небесными телами. В данном случае экстенсионал это планета Венера, а Утренняя и Вечерняя звезда - два различных интенсионала. В высказывании "взошла Утренняя звезда" можно свободно заменить Утреннюю звезду на Вечернюю и наоборот - смысл может поменяться, но истинностное значение - нет.

Экстенсиональность - известная черта классических логик. В интенсиональной же логике для определения истинностного значения некоторых высказываний недостаточно знать их форму и экстенсионалы компонентов. Иногда истинностное значение высказывания определяется "смыслом", невыразимым формально в рамках экстенсиональной логики. Как следствие, один или больше принципов субституции перестают быть валидными. Модальная логика является интенсиональной. Для того, чтобы более строго и полно осмыслить и систематизировать представления о свойствах модальной логики, требовался удобный инструмент, и таким инструментом стала идея о возможных мирах.

В любом возможном мире высказывание может быть истинным или ложным. Семантика возможных миров рассматривает модальные операторы как квантификаторы возможных миров, то есть, они перечисляют, в каких возможных мирах рассматриваемое модальное высказывание истинное, а в каких ложное. Например, истинное высказывание истинно в актуальном (нашем) мире. Ложное высказывание ложно в актуальном мире. Необходимо истинное высказывание (аналитическая истина/тавтология) истинно во всех возможных мирах. Возможное высказывание истинно как минимум в одном возможном мире. Невозможное высказывание (например, содержащее логическое противоречие) не истинно ни в одном мире. И т.д.

Существуют разные точки зрения на онтологический статус возможных миров, то есть, существуют ли они реально. Сторонники конкретизма (в частности, сам Льюис) считали, что да - множество возможных миров это множество физических контекстов, конкретных миров. Абстракционисты (Сталнакер и др.) придерживаются точки зрения, что возможные миры можно рассматривать как потенциальные состояния актуального мира. Крипке же считает, что возможные миры это чисто логические конструкты.


\section{Современные тенденции в философии математики}
\section{Структуралистский подход к обоснованию математики}
Современный структурализм, как и Н. Бурбаки, считает наиболее фундаментальным в математике понятие структуры, а не абстрактного объекта и даже множества, которое оказывается частным случаем структуры. С этой точки зрения, П. Бенацерафф, напр., выступает против представления чисел множествами или абстрактными объектами, рассматривая их как знаки (цифры) в определенной знаковой структуре. Остается, однако, онтологический вопрос: является ли понятие структуры более простым и удобным, чем абстрактный объект? Не ведет ли это к отождествлению математического знания с эмпирическим? Убедительного ответа на него современные структуралисты не дают
\section{Математическое объяснение. Математика как метафора}

\end{document}
