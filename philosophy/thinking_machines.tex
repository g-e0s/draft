\documentclass[12pt, specialist, subf, substylefile = spbu.rtx]{disser}


\usepackage[a4paper, includefoot,
            left=3cm, right=1.5cm,
            top=2cm, bottom=2cm,
            headsep=1cm, footskip=1cm]{geometry}

\usepackage[T1]{fontenc}
\usepackage[utf8]{inputenc}
\usepackage[english, russian]{babel}
\usepackage{moreverb}
\usepackage{array}

\setcounter{tocdepth}{2}

\graphicspath{{fig/}}

\newcommand{\Expect}{\mathsf{E}}
\newcommand{\projVec}{W}

\newtheorem{remark}{Замечание}

\begin{document}

\institution{%
    Министерство образования и науки Российской Федерации \\
    Федеральное агентство по образованию \\
    Федеральное государственное образовательное учреждение высшего
    профессионального образования «Санкт-Петербургский государственный университет» \\
    Математико-механический факультет \\
    Кафедра математической кибернетики
}

\apname{д.\,ф.-м.\,н., профессор А.\,Л.~Фрадков}
%\title{Реферат}
\title{Думающие машины}

%\topic{\normalfont\scshape %
%<<Думающие машины>>}

\author{Сарапулов Георгий Владимирович}

\sa {М.\,С.~Ананьевский}
\sastatus{к.\,ф.-м.\,н., доцент}

\city{Санкт-Петербург}
\date{2018}

\maketitle


\tableofcontents

\intro
Discussion is held around two thought experiments of Turing and Searle, prefigured earlier by Descartes and Leibnitz. According to Turing, we should consider a computer able to pass behavioral test as having intelligence, while Searle states that formal symbol manipulaitions are semantics-free and thus create no understanding. 

\cite{turing1950computing}
\cite{Searle80minds}

%\chapter{Вспомогательные результаты}

\chapter{Игра в имитацию}
\label{sec:turing}

В статье \cite{turing1950computing} Тьюринг отмечает, что ответ на вопрос <<Могут ли машины думать>> необходимо было бы начать с определения понятия <<машина>> и <<думать>>. Вместо этого, чтобы избежать неоднозначности толкований, связанных с повседневным пониманием этих слов, он заменяет исходный вопрос мысленным экспериментом, который он сам назвал <<Игрой в имитацию>> (<<The Imitation Game>>), впоследствии ставшим известным как <<тест Тьюринга>>. Суть первоначальной игры заключалась в том, что один из участников игры задает вопросы двум другим участникам (мужчине и женщине, находящимся в отдельной комнате) чтобы выяснить, кто из них двоих - мужчина, а кто - женщина. Один из опрашиваемых (A) старается обмануть опрашивающего, второй (B), наоборот, старается ему помочь. В рамках мысленного эксперимента Тьюринг заменяет участника A на машину и задает вопрос: будет ли опрашивающий ошибаться в новой игре между человеком и машиной так же часто, как он ошибался бы в игре между мужчиной и женщиной?

В такой игре предполагается, что лучшей стратегией для машины будет пытаться давать ответы, какие кажутся правдободными для ответов человека. Под машиной подразумевается цифровой компьютер с конечным числом состояний, который следует фиксированным инструкциям из своего хранилища, когда осуществляет действия и не имеет права от них отклоняться. 

%Машина может быть результатом любой инженерной технологии, даже такой, которую инженеры не в состоянии удовлетворительно описать из-за экспериментальности возможного метода. Из машины исключены люди, рожденные в обычной манере. Наличие датчика случайных чисел иногда описывается как свободная воля машины, но Тьюринг против. Обычно невозможно из наблюдения за машиной определить, есть ли у нее случайный элемент, т7 к. похожих эффектов можно добиться и детерминированно, например на основе цифр числа пи.

Сам Тьюринг считал, что с уже спустя полвека будет возможно запрограммировать компьютер для игры в имитацию на таком уровне, что опрашивающий будет верно идентифицировать собеседников в пятиминутном диалоге в среднем не более чем в 70\% случаев. Против такой точки зрения имеюется ряд возражений, которые Тьюринг последовательно рассматривает.

Согласно теологическому возражению, мышление порождено бессмертной душой, данной Богом каждому человеку, но не животному или машине, в связи с чем животные и машины мыслить не способны. Контраргумент: мусульмане считают, что у женщины нет души; оставаясь в границах принятой в теологии аргументации, Тьюринг считает, что приведенное выше возражение влечет существенное ограничение власти Всевышнего, который должен быть способен наделить душой как слона (наделив его более совершенным мозгом для обслуживания души), так и машину. При этом создание такой машины узурпировало бы исключительную власть бога на создание душ не в большей степени, чем рождение детей, так как в обоих случаях люди выступают как инструменты его воли, предоставляющие обитель для создаваемых им душ. Тьюринг, однако, отмечает спекулятивный характер подобных рассуждений и признает неудовлетворительность основанной на них аргументации.

Образно названное <<Головой в песке>> возражение относится к опасениям, связанных с возможными ужасными последствиями появления мышления у машин. Будучи связанным с первым возражением посредством принятия роли человека ка высшего существа, оно недостаточно существенно для опровержений.

Математические возражения опираются на ряд результатов в математической логике, в частности, теорему Гёделя о неполноте, которые показывают ограниченность возможностей дискретных машин. В качестве примера ограниченности машин при <<игре в имитацию>> Тьюринг указывает на тип вопросов, на которые цифровой компьютер с бесконечной памятью неспособен дать ответ: <<Представьте себе машину [описание машины]. Сможет ли она ответить \'Да\' на любой вопрос?>>. Математический результат показывает, что если описание машины в какой-то степени схоже с описанием опрашиваемой машины, ответ будет неверным либо его вообще не последует.






\chapter{Китайская комната}
\label{sec:chinese}

\begin{remark}
	$\alpha$
\end{remark}

~\ref{sec:chinese} 

\conclusion

\bibliography{bibliography}
\bibliographystyle{gost2008}
%\appendix
%\chapter{Приложение 1}
\end{document}
